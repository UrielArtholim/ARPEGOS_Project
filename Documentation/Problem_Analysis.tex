% Project Management Plan Documentation Template %
% Template made following ISO/IEC/IEEE 16326:2009 %

% Author : Alejandro Muñoz Del Álamo %
% Copyright 2019 %

% Problem Analysis %

\part{Desarrollo}
\chapter{Análisis del problema}
A la vista de lo expuesto previamente en el apartado 
\textit{Crítica al estado del arte}, se hace necesario el 
planteamiento de un análisis de la problemática razonada para 
poder esbozar una propuesta que permita resolver los aspectos 
descritos en el apartado anteriormente mencionado.\medskip

\section{Catálogo de actores}
A causa de que el sistema se define como una aplicación móvil, 
se alcanza la conclusión de sólo considerar un único actor en 
este proyecto, que es el usuario de la aplicación.

\section{Análisis de requisitos del sistema}
En la presente sección vamos a proceder a realizar un análisis de 
requisitos del sistema, que recoge y describe el conjunto 
de requisitos específicos del sistema que se va a desarrollar.\medskip

En primera instancia, se presentarán los requisitos agrupados en 
conjuntos funcionales del sistema. Posteriormente, se describirán 
los casos de uso en el próximo capítulo. \medskip

Para ello, se hará una diferenciación entre 
\textit{requisitos funcionales}, que son aquellos que detallan 
la funcionalidad del sistema, y \textit{requisitos no funcionales},
que refieren a otros aspectos del software que deben ser satisfechos.

\section{Requisitos funcionales}
Un requisito funcional especifica una función concreta del sistema o 
de alguno de sus componentes. A continuación se muestran los requisitos 
funcionales, estructurados según el módulo del sistema al que refieren.
%\begin{itemize}
    %\item \textbf{\textit{OBJ-001}}: \textbf{Almacenamiento de información}.
    %La aplicación debe disponer de una estructura que permita almacenar y 
    %clasificar la información contenida.
    %\begin{itemize}
        
    %    \item El sistema debe disponer de un conjunto de directorios, en el que 
    %    cada directorio hace referencia a un juego diferente.

    %    \item Cada directorio de juego estará a su vez compuesto por dos 
    %    directorios: uno para almacenar los ficheros referentes a la información 
    %    del juego (\textit{gamefiles}), y otro para almacenar los personajes generados con ese juego
    %    (\textit{characters}).

    %    \item El directorio \textit{gamefiles} contendrá un fichero en formato \textbf{OWL} 
    %    por cada versión accesible del juego en cuestión.
        
    %    \item El directorio \textit{characters} sólo contendrá los ficheros de personaje 
    %    generados por el sistema durante el proceso de creación de personaje.
        
    %\end{itemize}

    %\item \textbf{\textit{OBJ-001}}: \textbf{Lógica de la aplicación}.
    El sistema precisa de una lógica estructurada y compleja que permita procesar 
    información de diferentes fuentes, de manera que los procesos del sistema se 
    adecuen a su contenido.
    \begin{itemize}
        
        \item \textbf{\textit{OBJ-001}}: El usuario podrá seleccionar un juego concreto 
        (juego activo) para poder acceder a la información relacionada con el mismo.

        \item \textbf{\textit{OBJ-002}}: El usuario podrá crear un personaje para el juego activo,
        mediante un proceso guiado paso a paso
        
        \item \textbf{\textit{OBJ-003}}: El usuario podrá seleccionar un personaje (personaje activo) 
        de todos los existentes para el juego activo
        
        \item \textbf{\textit{OBJ-004}}: El usuario podrá visualizar la información del 
        personaje seleccionado del juego activo.
        
        \item \textbf{\textit{OBJ-005}}: El usuario podrá eliminar un personaje ya creado 
        del juego activo.
        
        \item \textbf{\textit{OBJ-006}}: El usuario podrá realizar cálculos con los valores 
        de las habilidades del personaje activo.

    \end{itemize}
%\end{itemize}


\section{Requisitos no funcionales}
Un requisito no funcional es una propiedad o cualidad que no forma parte de los 
fundamentos del sistema, pero es necesario para que el producto cumpla con su
cometido apropiadamente. \medskip

Para la declaración de requisitos no funcionales, se establecerán como base los 
requisitos indicados en las normas \textit{IEEE Std. 830} e \textit{ISO/IEC 25010 (SQuaRE)}:

\begin{itemize}
    
    \item \textbf{Adecuación funcional}: La aplicación debe cumplir con todos los requisitos necesarios, 
    de manera que sea completo y correcto funcionalmente.

    \item \textbf{Seguridad}: El sistema no requiere asegurar la información que procesa, debido a que 
    no contiene información sensible del usuario en ningún momento, ni realiza conexión externa alguna 
    para obtener información.
    
    \item \textbf{Compatibilidad}: La aplicación deberá ser compatible con los ficheros que contienen
    la información de los juegos que formarán parte del sistema.
    
    \item \textbf{Usabilidad}: El sistema debe disponer de una interfaz de usuario intuitiva y fácil de 
    manejar, de manera que pueda ser utilizado por usuarios sin conocimientos técnicos ni avanzados de 
    informática. La curva de aprendizaje deberá ser lo más reducida posible, de manera que personas de 
    cualquier ámbito puedan hacer uso del mismo.

    \item \textbf{Fiabilidad}: La aplicación deberá estar libre de errores que influyan negativamente 
    en su uso normal. Debido a que la aplicación depende de información incluida por terceros, será 
    necesario comprobar que dicha información es compatible con la aplicación.
    
    \item \textbf{Eficiencia}: El sistema debe evitar en la medida de lo posible el uso de información 
    redundante para poder asegurar su funcionamiento cuando se introduzcan juegos de alta complejidad 
    que requieran un elevado uso de recursos.

    \item \textbf{Mantenibilidad}: Este apartado representa la capacidad del producto software para 
    ser modificado efectiva y eficientemente. Esto será posible debido al desarrollo de código 
    limpio y bien documentado, al diseño y la implementación modular del mismo. Se plantea el uso 
    de patrones de arquitectura de software, tales como \textbf{MVVM}.

    \item \textbf{Portabilidad}: El sistema está diseñado para su uso en dispositivos móviles, 
    aunque no se descarta una futura ampliación para introducirlo en otro tipo de dispositivos.
    La implementación está realizada únicamente para sistemas \textit{Android}, ya que no 
    se dispone de las herramientas necesarias para el despliegue en \textit{Mac OS}. 

\end{itemize}

\section{Reglas de negocio}
Toda funcionalidad está relacionada con el único actor del proyecto, el usuario de la aplicación.

\section{Análisis de las soluciones}
Una vez se han indicado los requisitos del proyecto, es necesario considerar de qué maneras se puede 
realizar una aplicación que cumpla con dichas cláusulas. Para ello, se realizarán propuestas en base a 
las opciones disponibles, y tras realizar una comparativa, se seleccionará la mejor opción como la 
solución propuesta. \medskip

Hoy en día, la informática dispone de multitud de herramientas para realizar proyectos de diversa envergadura.
Como no se puede abarcar un conjunto tan amplio de posibilidades, se ha optado por seleccionar algunas de estas, 
en base a dos criterios: plataforma en la que se realiza la ejecución y el sistema de almacenamiento de la información 
de la aplicación. \medskip

En primer lugar, haremos referencia a las posibilidades seleccionadas en base al criterio de la plataforma en la que 
se realiza la ejecución de la aplicación: 
\begin{itemize}
    \item \textbf{Aplicación de escritorio}: Una aplicación de escritorio es aquella que se encuentra instalado en el 
    ordenador o sistema de almacenamiento (USB) y podemos ejecutarlo sin internet en nuestro sistema operativo, 
    al contrario que las aplicaciones en la nube que se encuentran en otro ordenador (servidor) al que accedemos 
    a través de la red o internet a su software.
    % https://es.wikipedia.org/wiki/Aplicación_de_escritorio

    \item \textbf{Aplicación Web}: Se denomina aplicación web a aquellas herramientas que los usuarios pueden utilizar 
    accediendo a un servidor web a través de internet o de una intranet mediante un navegador.
    % https://es.wikipedia.org/wiki/Aplicación_web 

    \item \textbf{Aplicación multiplataforma}: Se denomina multiplataforma a un atributo conferido a programas informáticos 
    o métodos y conceptos de cómputo que son implementados, y operan internamente en múltiples plataformas informáticas.
    % https://es.wikipedia.org/wiki/Multiplataforma

    %Buscar definiciones diferentes
\end{itemize}

En segundo lugar, se mostrarán las opciones basadas en el criterio del sistema de almacenamiento de la información:
\begin{itemize}
    \item \textbf{Sistema de bases de datos}: C \textit{Un sistema de bases de datos es una colección de datos almacenados conjuntamente
    con su descripción \textnormal{(base de datos)} y un sistema harware/software para la gestión de su fiabilidad, seguridad, 
    modificación y recuperación \textnormal{{(sistema gestor de bases de datos, SGBD)}}}. Los sistemas de bases de datos
    % Object-oriented Database Systems: the Notion and the Issues
    \item \textbf{Ontología}:
\end{itemize}

% TO DO
% Ask for HEEEEEEEEEEEEEEELP

\section{Solución propuesta}
\lorem()

\section{Análisis de seguridad}
En todo proyecto deben tenerse en cuenta todos los posibles accesos al sistema. Por 
esto mismo, se recomienda realizar un estudio de las conexiones al mismo, que permita reconocer las 
posibles vulnerabilidades que puedan darse a causa de éstas, y poder emplear las medidas necesarias 
para evitar dichas vulnerabilidades. \medskip

El proyecto actual no requiere ningún tipo de conexión externa o entre sistemas para poder operar, y 
debido a esto, no procede realizar este análisis, aunque no se descarta que deba realizarse tras 
incorporar alguna funcionalidad que las requiera.

\section{Análisis de la protección de datos}
Al igual que en el apartado \textit{Análisis de seguridad}, todos los proyectos deben tener constancia 
de la información que procesan y de la sensibilidad de la misma, con motivo de evitar una posible filtración 
de la misma. \medskip

Dada la naturaleza del proyecto, no hay información que requiera un control específico 
de seguridad, ya que toda la información disponible en el sistema, está disponible en los libros de los
juegos que se utilicen en la aplicación. Por tanto, no procede realizar un análisis de seguridad para 
el presente proyecto, aunque en caso de introducir información sensible en alguna mejora futura, este 
análisis debería realizarse para protegerla.

\section{Presupuesto}
% TO DO




