% ARPEGOS:  Automatized Roleplaying-game Profile Extensible Generator Ontology based System %
% Author : Alejandro Muñoz Del Álamo %
% Copyright 2019 %

% Section 9.2: Características %
\section{Gestión de Personajes}
\subsection{Añadir personaje}
Esta función permite al usuario crear un nuevo personaje siguiendo las bases del juego que esté seleccionado como activo.
En caso de que el proceso quede sin finalizar, habrá que iniciarlo desde cero.\medskip

\begin{enumerate}
    \item Acceder al menú principal y pulsar la opción \textit{Crear personaje}.
    \item Introducir el nombre del personaje. \textbf{Aviso}: \textit{El nombre de personaje debe ser único.}
    \item Seleccionar una de las opciones listadas en la \textit{etapa raíz} del proceso de creación.
    \item Interactuar con la vista que se muestre, según se requiera (seleccionar uno o varios elementos, o 
    introducir valores).
    \item Utilizar los botones con flechas direccionales para avanzar o retroceder de etapa.
    \item Realizar los pasos 4 y 5 con todas las vistas disponibles.
    \item Tras configurar el personaje a medida, pulsar el botón \textit{Finalizar} presente en la última 
    etapa para guardar la información del personaje, y habrá acabado el proceso de creación.
\end{enumerate}

\subsection{Visualizar y modificar personaje}
Esta función muestra al usuario la información del un personaje del juego selecionado como activo.\medskip

\begin{enumerate}
    \item Acceder al menú principal y pulsar la opción \textit{Visualizar personaje}. Al hacer esto, 
    se mostrará un listado de los diferentes personajes disponibles del juego activo.
    \item Elegir uno de los personajes disponibles y pulsar el botón \textit{Continuar}. 
    Una vez elegido, se mostrará un listado con las diferentes versiones del juego disponibles.
    \item Se muestra un carrusel con la información del personaje, mostrada por etapas (igual que en el proceso de creación).
    Para navegar por las diferentes vistas, utilizar los botones de las flechas de dirección.
    \item En caso de querer editar la información de la vista actual, presionar el botón \textit{Editar}, y se mostrará la 
    pantalla de edición de la etapa. 
    \item Tras realizar las modificaciones oportunas, pulsar el botón \textit{Aceptar} para sobreescribir la información del 
    personaje.
\end{enumerate}

\subsection{Eliminar Personaje}
Esta función permite eliminar un personaje de un juego. \medskip

\begin{enumerate}
    \item Acceder al menú principal y pulsar la opción \textit{Eliminar personaje}.
    \item Seleccionar uno de los juegos disponibles.
    \item Seleccionar uno de los personajes de la lista y pulsar el botón \textit{Eliminar}.
    \item Se mostrará una pregunta para confirmar la operación. En caso afirmativo, pulsar el botón \textit{Confirmar}.
    En caso contrario, presionar el botón \textit{Cancelar}.
    \item Tras confirmar la operación, se eliminará el personaje correspondiente.
\end{enumerate}
