% ARPEGOS:  Automatized Roleplaying-game Profile Extensible Generator Ontology based System %
% Author : Alejandro Muñoz Del Álamo %
% Copyright 2019 %

% Section 11.2: Lecciones aprendidas %
\section{Resultados}
Este proyecto nació con el objetivo de desarrollar una aplicación que posibilitara a los jugadores de rol 
disponer de una herramienta práctica e intuitiva que facilitara uno de los factores más complejos de los 
juegos de rol: el proceso de creación de personajes. \medskip 

La propuesta consistía en diseñar e implementar un sistema que permitiera generar y gestionar personajes de juegos de rol, 
y que también hiciera uso de la información de los personajes para agilizar los cálculos más repetidos 
durante una partida de rol. \medskip

En primer lugar se realizó un estudio del estado del arte, para observar algunas de las herramientas disponibles en el mercado, 
permitiendo al equipo de desarrollo convertir esta información en objetivos claros para la elaboración del proyecto.\medskip

Una vez aclarados los objetivos, comenzó la fase de planificación del proyecto, en la que se planificó y dividió el trabajo en 
una serie de iteraciones aplicando la metodología ágil \textit{Scrum}. Aunque la planificación fue planificada de manera cuidadosa, 
resultó imposible no acumular retraso en el desarrollo del proyecto.\medskip

Tras analizar los posibles riesgos y alternativas para la realización del proyecto, se optó por diseñar e implementar la aplicación 
móvil utilizando el \textit{framework Xamarin} en conjunto con \textit{ontologías} para almacenar la información de los juegos de rol.
Para el sistema de pruebas se decidió por (\textit{completar}). 

En lo referente al diseño de la aplicación, se decidió utilizar el patrón de arquitectura a tres capas \textit{MVVM}. 
Se diseñó un modelo lógico general para los ficheros de información de los juegos de rol y se realizaron bocetos de 
las interfaces de usuario. Llegados a este punto, se procedió a construir la aplicación y finalmente, se desarrollaron 
dos manuales: un manual de usuario para utilizar la aplicación, y un manual de desarrollo para crear ontologías que 
puedan ser utilizadas por la aplicación.\medskip 

Todos los objetivos propuestos al inicio del proyecto han sido cubiertos a la finalización del mismo. El mayor problema 
consistía en generalizar las funciones del sistema, de forma que la aplicación fuera completamente independiente de la 
información del juego, para así poder utilizar cualquier juego en el sistema. Además, para que la idea propuesta funcionara, el 
sistema tenía que ser genérico, pero debía ser capaz de trabajar con la totalidad de la información del juego, independientemente 
de su complejidad. \medskip 

Otro de los objetivos clave era disponer de una interfaz intuitiva que pudiera facilitar a los usuarios consultar información sobre 
cualquier elemento del juego.