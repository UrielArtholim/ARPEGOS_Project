% ARPEGOS:  Automatized Roleplaying-game Profile Extensible Generator Ontology based System %
% Author : Alejandro Muñoz Del Álamo %
% Copyright 2019 %

% Section 11.3: Trabajo futuro %
\section{Lecciones aprendidas}
Un proyecto de final de grado tiene como objetivo demostrar que el alumno que lo realiza es capaz de poner en práctica 
todo aquello que ha aprendido a lo largo de sus estudios, y además puede añadir valor al proyecto, haciendo que éste 
sea único y demuestre que el alumno merece la obtención de un título universitario. \medskip

Durante la elaboración de este proyecto, se ha demostrado la importancia de conocer las ambiciones del proyecto y 
su profundidad, pues resulta muy complicado hacer un análisis claro de éste si no entienden sus fundamentos. 
Además, ha resaltado la importancia de analizar correctamente qué se quiere conseguir, y de que forma, pues 
no siempre se dispone de las herramientas necesarias para realizar el desarrollo de la manera deseada. Asimismo, 
las fases de recolección de requisitos y diseño aportan garantías al proyecto, de manera que una modificación en 
la planificación no resulta un quebradero de cabeza si el trabajo se ha planteado correctamente. \medskip

El aprendizaje de herramientas desconocidas ha resultado también importante, pues la informática es un campo 
que tiene un avance exponencial y siempre está sujeto a cambios, por lo que es importante ser capaz de 
adoptar nuevos materiales y mecanismos para adaptarse a los avances.\medskip

Por último, pero no por ello menos importante, ha resultado impresionante como es posible que áreas del conocimiento 
tan diferentes a la informática, como la literatura o la filosofía puedan estar directamente relacionadas entre sí, 
y sirvan para seguir avanzando poco a poco en el vasto mundo del conocimiento.