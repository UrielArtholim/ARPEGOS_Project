% ARPEGOS:  Automatized Roleplaying-game Profile Extensible Generator Ontology based System %
% Author : Alejandro Muñoz Del Álamo %
% Copyright 2019 %

% Section 11.3: Trabajo futuro %
\section{Trabajo futuro}
Aunque se ha logrado cumplir con todos los objetivos propuestos para este proyecto, siempre es posible introducir nuevas 
funciones que mejoren el apartado técnico o visual del mismo. A continuación se proponen algunas ideas para añadir en 
un futuro al proyecto:

\begin{itemize}
    
    \item Montar un servidor que contenga todos los archivos de juegos disponibles para la aplicación, de manera 
    que se puedan añadir juegos a la aplicación desde el servidor. Aunque la aplicación no requiera conexión para funcionar, 
    sería interesante poder crear un repositorio de juegos para que los usuarios puedan disponer de ellos, lo que facilitaría 
    ademas la colaboración de los miembros de la comunidad.

    \item Como la aplicación está desarrollada en \textit{Xamarin}, que es un \textit{framework} preparado para hacer 
    desarrollo multiplataforma, no estaría de más poder aprovechar eso para llevar esta aplicación, que actualmente sólo 
    está disponible en \textit{Android} a sistemas \textit{iOS} y ordenadores.

    \item Hoy en día es normal que se lleven a cabo partidas de rol por videoconferencia, de manera que los jugadores 
    pueden llevar a cabo sus partidas sin necesidad de estar reunidos físicamente. Por tanto sería interesante incluir 
    un sistema que permita al director de la partida conocer la información de los personajes que participan en tiempo 
    real, de forma que pueda disponer de su información en todo momento.

    \item Sería muy interesante poder ampliar el sistema de cálculo de habilidades para poder realizar todos 
    los cálculos de combate, de manera que los jugadores podrían ahorrar tiempo para disfrutar más del apartado 
    interpretativo de las partidas.

    \item Tampoco estaría de más introducir una función que simule el lanzamiento de los dados, de tal 
    manera que la aplicación pueda realizar el lanzamiento y ofrecer el resultado total de la tirada.

    \item Hacer que el sistema desarrollado en este proyecto sea multilingüe, ya que jugar a juegos de rol es una 
    práctica extendida en todo el mundo, y permitiría abrir el proyecto a comunidades de otros países.

\end{itemize}