% Project Management Plan Documentation Template %
% Template made following ISO/IEC/IEEE 16326:2009 %

% Author : Alejandro Muñoz Del Álamo %
% Copyright 2019 %

% Preface File %

% Greeting Clause  % 

\thispagestyle{prefacepage}
{\large \textbf{\textit{Agradecimientos}}} 
\vspace{0.5cm} \\
\textit{Quiero dar las gracias a mis tutores Pablo y Alberto, pues no habría podido 
realizar este trabajo sin ellos. También quiero agradecer a mis padres y a mi hermano 
todo el tiempo y esfuerzo que han dedicado para ayudarme. No quiero olvidarme de agradecer a 
la editorial Ex Regnum Publishing, por permitirme utilizar parte de su material 
para la elaboración de la documentación de este proyecto.}

\blankpage{}

\thispagestyle{prefacepage}
{\large \textbf{\textit{Dedicatoria}}} 
\vspace{0.5cm} \\
\textit{A Marta, por todas las aventuras que hemos vivido, y las que nos quedan por vivir.}

\blankpage{}

\thispagestyle{prefacepage}
{\large \textbf{\textit{Resumen}}} 
\vspace{0.5cm} \\
La esencia del juego de rol consiste en la inmersión de los jugadores en un mundo diferente 
lleno de aventuras por vivir y lugares por explorar. Para ello, cada jugador requiere 
la existencia de un \textit{avatar} en el juego, que defina las características y habilidades del 
jugador en ese mundo. Este proyecto busca agilizar el proceso de creación de ese avatar, tanto para 
facilitar el acceso a este tipo de juegos a personas que no tengan experiencia previa, como a jugadores 
experimentados, permitiéndoles generar un personaje completo en el menor tiempo posible, sin la necesidad 
de conocer el funcionamiento del sistema del juego. 
\vspace{2cm}

{\large \textbf{\textit{Abstract}}} 
\vspace{0.5cm} \\
The essence of a roleplaying game lies in the inmersion of the players in a different world 
full of adventures to be lived and places to be explored. In order to achieve that, each player requires 
the existence of an \textit{avatar}, a character which defines the characteristics and skills of the player 
inside that world. This project looks forward to speed up the creation process of that avatar, to ease 
the access to this kind of game to people which have no previous experience, and also to seasoned players, 
allowing them to generate a full character in the shortest amount of time, without the need of know of 
the functioning of the game system. 
\vspace{2cm}

{\large \textbf{\textit{Palabras clave}}} 
\vspace{0.5cm} \\
\textit{juego de rol}, \textit{creación de personajes}, \textit{automatización},
\textit{ontologías}, \textit{web semántica}
\blankpage{}

% Documentation License Page %
%{\large \textbf{Licencia}}
%\vspace{0.5cm}
%\textit{Copyright} \textcopyright 2019 \autor, todos los derechos reservados.
%
%Las licencias de las imágenes de terceros se especifican en su uso. \newline
%
%Los logotipos de empresas e instituciones que aparecen en este
%documento son propiedad de dichas empresas e instituciones, y quedan bajo la licencia
%establecida por estas. \newline
%
%La plantilla de este documento ha sido creada por Alejandro Muñoz Del Álamo bajo 
%Licencia MIT (GNU \textit{Massachussetts Institute of Technology}) y con 
%\textit{copyright }\textcopyright  2019 Alejandro Muñoz Del Álamo, en los términos siguientes.
%{\itshape "Permission is hereby granted, free of charge, to any person obtaining a copy of 
%this software and associated documentation files (the "Software"), to deal in the Software 
%without restriction, including without limitation the rights to use, copy, modify, merge, 
%publish, distribute, sublicense, and/or sell copies of the Software, and to permit persons 
%to whom the Software is furnished to do so, subject to the following conditions:}
%
%{\itshape The above copyright notice and this permission notice shall be included in all copies or 
%substantial portions of the Software."}

\blankpage{}

% Notation & Format Page % 
\thispagestyle{prefacepage}
{\large \textbf{Notación y formato}} \smallskip 
\vspace{0.5cm}\\
En la siguiente tabla se presenta un conjunto de convenios de notación de sintaxis.\smallskip 
\vspace{0.5cm}
\begin{center}
    \begin{tabular}{|l|l|} \hline
        \multicolumn{2}{|c|}{Notación establecida} \\ \hline
        \textbf{negrita} & Título o texto destacado \\ \hline
        \textit{cursiva} & Texto en otro idioma, destacado, \\ & citas o nombres de aplicaciones \\ \hline
        \texttt{monoespaciado} & Referencias a código fuente \\ \hline
        \underline{subrayado} & Advertencia para el lector \\ \hline
        {\color{Red}{color}} &  Enlace interno ({\color{Red} rojo}) \\ \hline
    \end{tabular}
\end{center}


\blankpage{}

% Abstract Page %


