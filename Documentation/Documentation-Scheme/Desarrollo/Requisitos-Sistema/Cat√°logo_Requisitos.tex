% ARPEGOS:  Automatized Roleplaying-game Profile Extensible Generator Ontology based System %
% Author : Alejandro Muñoz Del Álamo %
% Copyright 2019 %

% Section 3.3: Catálogo de Requisitos %
\section{Catálogo de requisitos} \label{Requisitos}
Como dice Távora \autocite*{Tavora2014}, un requisito es una cláusula 
que debe cumplir una aplicación de manera que se logre un objetivo 
o se solvente algún problema. \medskip

En la presente sección se procederá a realizar un análisis de 
requisitos del sistema, que recoge y describe el conjunto 
de requisitos específicos del sistema que se va a desarrollar.
Posteriormente, se describirán los casos de uso en el próximo capítulo. \medskip

Para catalogar los requisitos del sistema, se hará una diferenciación entre 
\textit{requisitos funcionales}, que son aquellos que detallan 
la funcionalidad del sistema, y \textit{requisitos no funcionales},
que refieren a otros aspectos del software que deben ser satisfechos.

\subsection{Requisitos funcionales} \label{Requisitos_funcionales}
Citando a Wiegers y Beatty \autocite*{Wiegers2013}, \textit{“los requisitos funcionales 
detallan los comportamientos observables que el sistema mostrará bajo ciertas condiciones 
y las acciones que el sistema dejará tomar a los usuarios.”} A continuación se muestran 
los requisitos funcionales de este proyecto \medskip

\begin{table}[htb]
    \begin{tabular}{|c|l|}
    \hline
    \thead{\textit{\textbf{Código}}} & \thead{\textit{\textbf{Requisito}}} \\ \hline \hline
    \makecell{\textbf{RF-01}} & \makecell{El sistema debe permitir al usuario seleccionar un juego del \\ 
    conjunto de juegos incluidos.} \\ \hline
    \makecell{\textbf{RF-02}} & \makecell{El sistema debe permitir al usuario crear paso a paso un \\
    personaje para el juego seleccionado.} \\ \hline
    \makecell{\textbf{RF-03}} & \makecell{El sistema debe adaptar el proceso de creación de personajes \\
    al juego seleccionado, sin importar su complejidad.} \\ \hline
    \makecell{\textbf{RF-04}} & \makecell{El sistema debe permitir la visualización de la información de \\
    un personaje creado por el usuario.} \\ \hline
    \makecell{\textbf{RF-05}} & \makecell{El sistema debe permitir la modificación parcial o total de la \\
    información de un personaje creado por el usuario.} \\ \hline
    \makecell{\textbf{RF-06}} & \makecell{El sistema debe permitir la eliminación de un personaje \\
    creado por el usuario} \\ \hline
    \makecell{\textbf{RF-07}} & \makecell{El sistema debe permitir realizar cálculos con la información de \\
    un personaje creado por el usuario.} \\ \hline
    \end{tabular}
    \caption{Tabla de requisitos funcionales}
    \label{Req_funcionales}
\end{table}

\subsection{Requisitos no funcionales}
Los usuarios pueden tener ideas sobre cómo funcionará la aplicación, 
tales como la fiabilidad, la rápidez de ejecución, la facilidad de uso, etc.
Estos aspectos quedan definidos por los \textit{requisitos no funcionales}, como expresan 
Stellman y Greene \autocite*{Stellman2005}.
\medskip

Para la declaración de requisitos no funcionales, se establecerán como base los 
requisitos indicados en las normas \textit{IEEE Std. 830} e \textit{ISO/IEC 25010 (SQuaRE)}:

\begin{itemize}   
    \item \textbf{Adecuación funcional}: La aplicación debe cumplir con todos los requisitos necesarios, 
    de manera que sea completo y correcto funcionalmente.

    \item \textbf{Seguridad}: El sistema no requiere asegurar la información que procesa, debido a que 
    no contiene información sensible del usuario en ningún momento, ni realiza conexión externa alguna 
    para obtener información.
    
    \item \textbf{Compatibilidad}: La aplicación deberá ser compatible con los ficheros que contienen
    la información de los juegos que formarán parte del sistema.
    
    \item \textbf{Usabilidad}: El sistema debe disponer de una interfaz de usuario intuitiva y fácil de 
    manejar, de manera que pueda ser utilizado por usuarios sin conocimientos técnicos ni avanzados de 
    informática. La curva de aprendizaje deberá ser lo más reducida posible, de manera que personas de 
    cualquier ámbito puedan hacer uso del mismo.

    \item \textbf{Fiabilidad}: La aplicación deberá estar libre de errores que influyan negativamente 
    en su uso normal. Debido a que la aplicación depende de información incluida por terceros, será 
    necesario comprobar que dicha información es compatible con la aplicación.
    
    \item \textbf{Eficiencia}: El sistema debe evitar en la medida de lo posible el uso de información 
    redundante para poder asegurar su funcionamiento cuando se introduzcan juegos de alta complejidad 
    que requieran un elevado uso de recursos.

    \item \textbf{Mantenibilidad}: Este apartado representa la capacidad del producto software para 
    ser modificado efectiva y eficientemente. Esto será posible debido al desarrollo de código 
    limpio y bien documentado, al diseño y la implementación modular del mismo. Se plantea el uso 
    de patrones de arquitectura de software, tales como \textbf{MVVM}.

    \item \textbf{Portabilidad}: El sistema está diseñado para su uso en dispositivos móviles, 
    aunque no se descarta una futura ampliación para introducirlo en otro tipo de dispositivos.
    La implementación está realizada únicamente para sistemas \textit{Android}, ya que no 
    se dispone de las herramientas necesarias para el despliegue en \textit{Mac OS}. 
\end{itemize}
