% ARPEGOS:  Automatized Roleplaying-game Profile Extensible Generator Ontology based System %
% Author : Alejandro Muñoz Del Álamo %
% Copyright 2019 %

% Section 3.4: Alternativas de Solución %

\section{Alternativas de Solución}
En esta sección, se ofrece un estudio del arte de las diferentes alternativas tecnológicas que permitan satisfacer 
los requerimientos del sistema, para optar por una de las opciones planteadas, que será dispuesta como base 
para el software a desarrollar.\medskip

Con este motivo, hemos optado por recoger algunas de las tecnologías existentes para realizar desarrollo de aplicaciones 
móviles.

\subsection{Entorno de desarrollo integrado}
Se conoce por \textit{Entorno de Desarrollo Integrado} o \textit{IDE} a las aplicaciones visuales que se utilizan para 
la elaboración de aplicaciones a partir de componentes, según explican Fuentes, Troya y Vallecillo. \autocite*{Fuentes}
Estas aplicaciones suelen disponer de ciertos elementos comunes, tales como:
\newpage
\begin{itemize}
    \item Colecciones que muestran los componentes disponibles mediante iconos.
    \item Un elemento central en el que se posicionan los componentes y se interrelacionan entre sí.
    \item Editores específicos para personalizar los componentes.
    \item Buscadores para localizar componentes.
    \item Directorios de componentes.
    \item Capacidad de desarrollo de nuevos componentes mediante el uso de los componentes ya disponibles.
    \item Acceso a herramientas de control y gestión de proyectos.
\end{itemize}

A continuación, mostramos los \textit{IDEs} que se han tomado en consideración para el desarrollo del presente proyecto.

\subsubsection{Android Studio}
Citando la página oficial, \textit{“Android Studio es el entorno de desarrollo integrado (\textnormal{IDE}) oficial para el desarrollo 
de apps para Android, basado en \textnormal{IntelliJ IDEA}.”} \autocite*{AndroidStudio2019}
Disponible para las plataformas \textit{Windows}, \textit{macOS} y \textit{GNU/Linux}. Basado en el lenguanje \textit{Java},
no tiene herramientas nativas para trabajar directamente con \textit{RDF} y \textit{OWL}. Para suplir este obstáculo, 
haríamos uso del framework libre \textit{Apache Jena}, cuya API permite trabajar con RDF, consiguiendo vincular 
el desarrollo en aplicaciones móviles con el uso de ontologías.

\begin{figure}[H]
    \centering
    \begin{minipage}{5cm}
        \centering
        \includegraphics[width=5cm]{Images/Logo_Android_Studio.png}
        \caption{Logo de \textit{Android Studio}}  
    \end{minipage}
    \hfill
    \begin{minipage}{5cm}
        \centering
        \includegraphics[width=5cm]{Images/Logo_Jena.png}
        \caption{Logo de \textit{Apache Jena}}  
    \end{minipage}
\end{figure}
% https://es.wikipedia.org/wiki/Android_Studio
% https://jena.apache.org/
\newpage
\subsubsection{React Native}
Según indica Wikipedia \autocite*{ReactNative}, \textit{“React Native es un \textit{framework} para el desarrollo de aplicaciones móviles de código abierto desarrollado por 
\textit{Facebook}. Se utiliza para desarrollar aplicaciones para \textit{Android}, \textit{iOS}, \textit{Web} y 
\textit{U} permitiendo a los desarrolladores usar \textit{React} con funcionalidades nativas de las plataformas.”}
Al igual que \textit{Android Studio}, React Native no tiene herramientas nativas para el desarrollo de ontologías, de 
manera que haríamos uso de bibliotecas tales como \textit{rdflib.js} para poder proceder al tratamiento de las ontologías.

\begin{figure}[H]
    \centering
    \begin{minipage}{5cm}
        \centering
        \includegraphics[width=5cm]{Images/Logo_React.jpeg}
        \caption{Logo de \textit{React Native}}  
    \end{minipage}
    \hfill
    \begin{minipage}{5cm}
        \centering
        \includegraphics[width=4cm]{Images/Logo_Rdflib.jpeg}
        \caption{Logo de \textit{rdflib.js}}  
    \end{minipage}
\end{figure}
% https://en.wikipedia.org/wiki/React_Native
% https://github.com/linkeddata/rdflib.js

\subsubsection{Xamarin}
Microsoft define Xamarin como una plataforma de código abierto para compilar aplicaciones modernas para \textit{iOS}, 
\textit{Android} y \textit{Windows} con \textit{\dotnet} \autocite*{Xamarin}. Xamarin es una capa de abstracción que administra la comunicación 
de código compartido con el código de plataforma subyacente. Xamarin dispone de varias bibliotecas que permiten trabajar con 
el formato \textit{RDF}, como pueden ser \textbf{dotNetRDF}, \textbf{FSharp.RDF}, \textbf{LITEQRDF} o \textbf{RdfMapperNet}, 
pero muy pocas trabajan con el formato \textit{OWL}, lo que hace que resulte fácil realizar un filtro de entre todas las 
posibilidades existentes, y optemos por la biblioteca \textit{\textbf{RDFSharp}}, creada por Marco de Salvo, que permite 
trabajar con ambos formatos, necesarios para el desarrollo de nuestro modelo ontológico.

\begin{figure}[H]
    \centering
    \includegraphics[width=5cm]{Images/Logo_Xamarin.png}
    \caption{Logo de \textit{Xamarin}}
\end{figure}
% https://docs.microsoft.com/es-es/xamarin/get-started/what-is-xamarin
% https://github.com/mdesalvo/RDFSharp
