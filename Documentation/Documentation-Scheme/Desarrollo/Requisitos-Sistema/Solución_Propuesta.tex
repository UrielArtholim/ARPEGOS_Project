% ARPEGOS:  Automatized Roleplaying-game Profile Extensible Generator Ontology based System %
% Author : Alejandro Muñoz Del Álamo %
% Copyright 2019 %

% Section 3.5: Solución Propuesta %
\section{Solución propuesta}
Tras considerar las opciones planteadas en el apartado anterior, se ha considerado descartar \textit{Android Studio} en 
primer lugar, ya que sólo permite el desarrollo en \textit{Android}, mientras que las otras soluciones permiten realizar 
el desarrollo en varias plataformas. \medskip

Una vez desechada una de las opciones, se ha comprobado que las soluciones restantes son compatibles con el proyecto, y 
prácticamente generan el mismo resultado. Por ello, en vez de considerar las plataformas, se ha realizado una comparación 
en función al lenguaje de programación con el que se trabaja en cada una de ellas, que son \textbf{JavaScript} en 
\textit{React Native}, y \textbf{C\#} para \textit{Xamarin}. Esta comparativa se realizará en forma de tabla.\bigskip

\begin{table}[htb]
\centering
\caption{Comparativa entre \textit{JavaScript} y \textit{C\#}}
\bigskip
\begin{tabular}{|c|c|c|}
    \hline
    & \textit{\textbf{JavaScript}} & \textbf{\textit{C\#}} \\ \hline \hline
    \textbf{Tipo de Lenguaje} & Scripting & Orientado a Objetos \\ \hline
    \textbf{Tipado} & Débil & Fuerte \\ \hline
    \textbf{Detección de errores} & Ejecución & Compilación y ejecución \\ \hline
    \textbf{Compilación} & No & Sí \\ \hline
    \textbf{Mantenibilidad} & Complejo & Sencillo \\ \hline
    \textbf{Soporte de IDE} & No & Microsoft Visual Studio \\ \hline
    \textbf{Sintaxis} & OBSL & OOP \\ \hline
\end{tabular}
\end{table}   

En la comparativa se puede observar que JavaScript es un lenguaje de scripting débilmente tipado que 
no requiere ser compilado, pero resulta difícil de mantener en sistemas complejos. Por otro lado, 
C\# es un lenguaje orientado a objetos fuertemente tipado que requiere ser compilado, que permite 
una mayor facilidad a la hora de mantener el código en sistemas de alta complejidad.\medskip

De entre estas dos posibilidades, el equipo de desarrollo ha tomado la decisión de que la mejor herramienta 
para la elaboración del presente proyecto es \textit{\textbf{Xamarin}} con la biblioteca \textit{\textbf{RDFSharp}},
ya que el C\# es un lenguaje orientado a objeto, y este paradigma es bastante similar a la estructura del modelo 
ontológico, que será uno de los pilares de la aplicación. Además, que tenga una mantenibilidad sencilla de comprender 
se compatibiliza con la idea de que cualquier usuario con los conocimientos suficientes pueda realizar ampliaciones 
en su funcionalidad. 

