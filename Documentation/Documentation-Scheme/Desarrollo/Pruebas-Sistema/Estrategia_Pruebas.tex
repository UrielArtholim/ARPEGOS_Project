% ARPEGOS:  Automatized Roleplaying-game Profile Extensible Generator Ontology based System %
% Author : Alejandro Muñoz Del Álamo %
% Copyright 2019 %

% Section 8.1: Introducción %
\section{Estrategia de pruebas}
Durante el desarrollo de cualquier proyecto de software, uno de los apartados fundamentales debe ser 
la realización de pruebas, que permitan conocer si el funcionamiento del sistema es correcto, o tiene 
errores que deben ser solventados. En un sistema de pequeña envergadura, es posible que sea simple y 
no requiera mucho esfuerzo, pero en sistemas más complejos no debe ser tomado a la ligera, pues 
se necesita un análisis previo de todos los elementos que componen el sistema, para diseñar un 
conjunto de pruebas que corroboren el correcto funcionamiento de dichos elementos, de manera individual y 
del sistema en conjunto. \newpage

A la hora de diseñar las pruebas, es conveniente planificar su desarrollo mediante una estrategia, 
que permita a los desarrolladores encargados desempeñar su trabajo de manera organizada y lógica.
La estrategia que se ha seguido para este proyecto ha sido la siguiente:\medskip

En primer lugar, se lleva a cabo la realización de las pruebas unitarias, ya que inspeccionan 
el correcto funcionamiento de los componentes de manera individualizada. Estas pruebas mejoran su 
eficacia si las clases están débilmente acopladas, puesto que de esta manera es posible verificar 
únicamente aquello que se desea, sin tener en cuenta otras clases distintas a las que se comprueban.\medskip

El próximo paso trata de ejecutar las pruebas de integración, que comprueba que el ensamblado de los 
componentes del sistema. Esta fase comienza cuando los diferentes componentes se agrupan para realizar 
funciones más complejas, que no pueden realizarse de manera individual por los componentes. \medskip

La tercera fase esta conformada por las pruebas de validación, en la que se revisan los requisitos
establecidos con la funcionalidad del sistema y cuyo éxito garantiza que el software cumple con todos 
los requerimientos establecidos en la sección \ref*{Requisitos}: \textit{Catálogo de Requisitos}. \medskip

Finalmente queda comprobar que el sistema desarrollado pueda ser utilizado en un entorno de explotación, 
ya sea para su ejecución con otros elementos, o su uso por parte del usuario final, lo que se realiza 
mediante las pruebas de sistema.