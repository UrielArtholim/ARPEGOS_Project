% ARPEGOS:  Automatized Roleplaying-game Profile Extensible Generator Ontology based System %
% Author : Alejandro Muñoz Del Álamo %
% Copyright 2019 %

% Section 8.1: Introducción %
\section{Pruebas Unitarias}
\subsection{Proyecto ARPEGOS Unit Test}
Para poder realizar pruebas utilizando \textit{Nunit}, es necesario añadir un nuevo proyecto a la 
solución, al cual hemos llamado \textit{ARPEGOS Unit Test}. Este proyecto se divide en dos partes 
claramente diferenciadas: la clase \textit{Setup}, que contiene la configuración necesaria para la 
ejecución de las pruebas, y las clases \textit{CharacterOntologyServiceTests} y 
\textit{OntologyServiceTests}, en las que están definidas las pruebas. Todas las funciones de prueba están 
nombradas de la siguiente manera:\medskip

\lbrack\textit{nombre\textunderscore función}\rbrack\textbf{\textunderscore Test}


\subsection{La clase \textit{Setup}}
Esta clase está compuesta por un conjunto de atributos privados, cuyo valor puede apreciarse 
mediante el uso de métodos observadores, y tres funciones: \textit{SetBaseFolder()},  
\textit{SetDefaultBaseFolder()} y \textit{Init()}. Las dos primeras reconfiguran las rutas de 
prueba, de manera que las pruebas puedan ejecutarse en una ruta local en vez de en un 
dispositivo Android. La función restante, es la que inicializa los valores de los distintos 
atributos de la clase, que pueden ser requeridos por algunas pruebas para su correcto funcionamiento.

\subsection{La clase \textit{OntologyServiceTests}}
La clase \textit{OntologyServiceTests} contiene las pruebas que comprueban la correcta ejecución 
de los métodos y funciones de la clase \textit{OntologyService}, que permiten 
crear, eliminar y cargar ontologías en la aplicación. Sus nombres y funciones quedan descritos en 
la tabla \ref{OntologyServiceTests}.

\begin{table}[H]
    \centering
    \begin{tabular}{|l|l|}
        \hline
        \thead{\textit{\textbf{Nombre de la función}}} &\thead{\textit{\textbf{Descripción}}} \\ \hline \hline
        \makecell[l]{\textit{LoadGame\textunderscore Test}} & \makecell[l]{Comprueba que la ontología del juego indicado\\ se carga correctamente.}\\ \hline
        \makecell[l]{\textit{LoadCharacter\textunderscore Test}} & \makecell[l]{Comprueba que la ontología del personaje indicado\\ se carga correctamente.}\\ \hline
        \makecell[l]{\textit{CreateCharacter\textunderscore Test}} & \makecell[l]{Comprueba que se genera un nuevo personaje\\ correctamente}\\ \hline
        \makecell[l]{\textit{DeleteCharacter\textunderscore Test}} & \makecell[l]{Comprueba que se elimina el personaje indicado\\ correctamente. }\\ \hline 
    \end{tabular}\medskip
    \caption{Pruebas unitarias. Clase \textit{OntologyServiceTests}}
    \label{OntologyServiceTests}
\end{table}
\justify
\subsection{La clase \textit{CharacterOntologyServiceTests}}
Las pruebas contenidas en esta clase prueban que todas las funciones relativas a la lectura o escritura 
de ontologías de personaje cumplen su objetivo correctamente. Dichas funciones quedan descritas en 
la tabla \ref*{CharacterOntologyServiceTests}.
\centering
\begin{longtable}{|l|l|}
    \hline
    \thead{\textit{\textbf{Nombre de la función}}} &\thead{\textit{\textbf{Descripción}}} \\ \hline \hline
    \endfirsthead
    
    \hline 
    \thead{\textit{\textbf{Nombre de la función}}} &\thead{\textit{\textbf{Descripción}}} \\ \hline \hline
    \endhead
    
    \hline
    \addlinespace \addlinespace
    \multicolumn{2}{c}{Sigue en la página siguiente.}
    \endfoot
    
    \addlinespace \addlinespace
    \caption{Pruebas unitarias. Clase \textit{OntologyServiceTests}}
    \label{CharacterOntologyServiceTests}
    \endlastfoot
    
    \makecell[l]{\textit{AddObjectProperty\textunderscore Test}} & \makecell[l]{Comprueba que se añade una propiedad de\\ objeto correctamente.}\\ \hline
    \makecell[l]{\textit{AddDatatypeProperty\textunderscore Test}} & \makecell[l]{Comprueba que se añade una propiedad de\\ datos correctamente.}\\ \hline
    \makecell[l]{\textit{AddClassification\textunderscore Test}} & \makecell[l]{Comprueba que se clasifica una propiedad \\ correctamente.}\\ \hline
    \makecell[l]{\textit{CheckLiteral\textunderscore Test}} & \makecell[l]{Comprueba que se valida un literal \\ correctamente.}\\ \hline
    \makecell[l]{\textit{CheckFact\textunderscore Test}} & \makecell[l]{Comprueba que se valida un hecho \\ correctamente.}\\ \hline
    \makecell[l]{\textit{CheckClass\textunderscore Test}} & \makecell[l]{Comprueba que se valida una clase \\ correctamente.}\\ \hline
    \makecell[l]{\textit{CheckObjectProperty\textunderscore Test}} & \makecell[l]{Comprueba que se valida una propiedad\\ de objeto correctamente.}\\ \hline
    \makecell[l]{\textit{CheckDatatypeProperty\textunderscore Test}} & \makecell[l]{Comprueba que se valida una propiedad\\ de datos correctamente.}\\ \hline
    \makecell[l]{\textit{CheckIndividual\textunderscore Test}} & \makecell[l]{Comprueba que se valida un individuo \\ correctamente.}\\ \hline
    \makecell[l]{\textit{CheckGeneralCost\textunderscore Test}} & \makecell[l]{Comprueba que se obtiene un coste\\ general correctamente.}\\ \hline
    \makecell[l]{\textit{CheckValueListInfo\textunderscore Test}} & \makecell[l]{Comprueba que se obtiene una lista\\ informativa de valores correctamente.}\\ \hline
    \makecell[l]{\textit{CheckAvailableOptions\textunderscore Test}} & \makecell[l]{Comprueba que se obtiene una lista\\ de opciones disponibles correctamente.}\\ \hline
    \makecell[l]{\textit{GetPropertyVisualizationPosition\textunderscore Test}} & \makecell[l]{Comprueba que se obtiene una posición\\ de visualización correctamente.}\\ \hline
    \makecell[l]{\textit{GetCharacterProperties\textunderscore Test}} & \makecell[l]{Comprueba que se obtiene la lista\\ de propiedades del personaje correctamente.}\\ \hline
    \makecell[l]{\textit{GetCharacterSkills\textunderscore Test}} & \makecell[l]{Comprueba que se obtiene la lista\\ de habilidades del personaje correctamente.}\\ \hline
    \makecell[l]{\textit{GetSkillValue\textunderscore Test}} & \makecell[l]{Comprueba que se obtiene el valor\\ de una habilidad correctamente.}\\ \hline
    \makecell[l]{\textit{GetElementClass\textunderscore Test}} & \makecell[l]{Comprueba que se obtiene la clase\\ de un elemento correctamente.}\\ \hline
    \makecell[l]{\textit{GetElementDescription\textunderscore Test}} & \makecell[l]{Comprueba que se obtiene la descripción\\ de un elemento correctamente.}\\ \hline
    \makecell[l]{\textit{GetIndividuals\textunderscore Test}} & \makecell[l]{Comprueba que se obtiene la lista\\ de elementos de una clase correctamente.}\\ \hline
    \makecell[l]{\textit{GetIndividualsGrouped\textunderscore Test}} & \makecell[l]{Comprueba que se obtienen la lista\\ de elementos agrupados de una clase\\ correctamente }\\ \hline
    \makecell[l]{\textit{GetAvailablePoints\textunderscore Test}} & \makecell[l]{Comprueba que se obtienen los puntos\\ disponibles para gastar correctamente.}\\ \hline
    \makecell[l]{\textit{GetLimit\textunderscore Test}} & \makecell[l]{Comprueba que se obtiene la propiedad\\ límite de una etapa de creación\\ correctamente.}\\ \hline
    \makecell[l]{\textit{GetStep\textunderscore Test}} & \makecell[l]{Comprueba que se obtiene el coste\\ unitario de una habilidad correctamente.}\\ \hline
    \makecell[l]{\textit{GetLimitValue\textunderscore Test}} & \makecell[l]{Comprueba que se obtiene el valor\\ de una propiedad límite correctamente.}\\ \hline
    \makecell[l]{\textit{GetObjectPropertyAssociated\textunderscore Test}} & \makecell[l]{Comprueba que se obtiene la propiedad\\ de objeto asociada a un elemento\\ correctamente.}\\ \hline
    \makecell[l]{\textit{GetOrderedSubstages\textunderscore Test}} & \makecell[l]{Comprueba que se pbtiene una lista\\ ordenada de etapas correctamente.}\\ \hline
    \makecell[l]{\textit{GetParentClasses\textunderscore Test}} & \makecell[l]{Comprueba que se obtiene una lista\\ de clases superiores en la jerarquía\\ correctamente.}\\ \hline
    \makecell[l]{\textit{GetGeneralCost\textunderscore Test}} & \makecell[l]{Comprueba que se obtiene el coste\\ general de una etapa correctamente.}\\ \hline
    \makecell[l]{\textit{GetPartialCost\textunderscore Test}} & \makecell[l]{Comprueba que se obtiene el coste\\ parcial de una etapa correctamente.}\\ \hline
    \makecell[l]{\textit{GetSubClasses\textunderscore Test}} & \makecell[l]{Comprueba que se obtiene una lista\\ de subclases correctamente.}\\ \hline
    \makecell[l]{\textit{GetDatatypeUri\textunderscore Test}} & \makecell[l]{Comprueba que se obtiene la URI\\ de un tipo de dato correctamente.}\\ \hline
    \makecell[l]{\textit{GetString\textunderscore Test}} & \makecell[l]{Comprueba que se obtiene la URI\\ de un elemento correctamente.}\\ \hline
    \makecell[l]{\textit{GetValue\textunderscore Test}} & \makecell[l]{Comprueba que se obtiene el valor\\ de un elemento dada su fórmula correctamente.}\\ \hline
    \makecell[l]{\textit{UpdateObjectAssertion\textunderscore Test}} & \makecell[l]{Comprueba que se actualiza una aserción\\ de objeto correctamente.}\\ \hline
    \makecell[l]{\textit{UpdateDatatypeAssertion\textunderscore Test}} & \makecell[l]{Comprueba que se actualiza una aserción\\ de datos correctamente.}\\ \hline
\end{longtable}
\justify