% ARPEGOS:  Automatized Roleplaying-game Profile Extensible Generator Ontology based System %
% Author : Alejandro Muñoz Del Álamo %
% Copyright 2019 %

% Section 6.1: Entorno de construcción %
\section{Entorno de construcción}
En esta sección se explican de qué herramientas se van a utilizar para la consecución de este proyecto.

\subsection{Framework}
Este proyecto hará uso de \textit{Xamarin}, un \textit{framework} para el desarrollo de aplicaciones multiplataforma, que 
utiliza \textit{C\#} como lenguaje de programación. \textit{Xamarin} posibilita de que dichas aplicaciones 
compartan el código base mediante \textit{Xamarin.Forms}, que permite crear interfaces de usuario en \textit{XAML} con 
código subyacente en \textit{C\#}, las cuales son representadas como controles nativos para cada plataforma.

\subsection{Entorno de desarrollo integrado}
El entorno de desarrollo integrado a utilizar es \textit{Visual Studio}, en su versión gratuita más actualizada hasta el momento
(\textit{Community 2019}). A continuación se muestran algunos de los motivos que impulsan al equipo de desarrollo a optar por este IDE:
\newpage
\begin{itemize}
    \item Soporte oficial de Xamarin para \textit{Visual Studio}
    \item Biblioteca \textit{RDFSharp} disponible mediante el administrador de paquetes \textit{Nuget}.
    \item Disponibilidad de la herramienta \textit{IntelliSense}, que permite una mayor maniobrabilidad con el código.
\end{itemize}

\begin{figure}[H]
    \centering
    \includegraphics[scale=0.2]{Figures/Logo_VisualStudio.png}
    \caption{Logo de \textit{Visual Studio}}
    \label{Logo_VS}
\end{figure}

\subsection{Editor de ontologías}
\label{Editor_ontologias}
Para la construcción de ontologías se ha utilizado el editor 
\protege en su versión \textit{5.5.0}.

\begin{figure}[H]
    \centering
    \begin{minipage}{0.38\textwidth}
        \centering
        \includegraphics[width=0.4\textwidth]{Figures/Logo_Protege.pdf}
        \caption{Logo de \protege}
    \end{minipage}
\end{figure}

\subsection{Control de versiones}
Se conoce como \textit{control de versiones} a la gestión de las modificaciones que se producen en los componentes de un producto, o en 
su configuración. Aunque puede hacerse manualmente, hoy en día existen herramientas que hacen este proceso más sencillo, cómodo y 
seguro. Estas herramientas se conocen como \textit{sistemas de control de versiones (\textbf{VCS})}.\medskip

Para realizar el control de versiones de este proyecto, se hará uso del VCS \textit{Git}. Además se utilizará la 
plataforma de desarrollo colaborativo \textit{GitHub} como repositorio remoto del proyecto. Así, en caso de que 
se corrompa la versión local del proyecto, será posible recuperar la versión más actualizada del proyecto.
\newpage
% Insertar imagen Git & GitHub
\begin{figure}[H]
    \centering
    \begin{minipage}{0.38\textwidth}
        \centering
        \includegraphics[width=0.5\textwidth]{Figures/Logo_Git.png}
        \caption{Logo de \textit{Git}}
    \end{minipage} \hspace{2cm}
    \begin{minipage}{0.38\textwidth}
        \centering
        \includegraphics[width=0.8\textwidth]{Figures/Logo_GitHub.jpg}
        \caption{Logo de \textit{GitHub}}
    \end{minipage}
\end{figure}

\subsection{Memoria}
En lo referente a la elaboración de la memoria, se ha realizado en \LaTeX, utilizando \textit{Visual Studio Code} como editor 
de texto, con la extensión \textit{\LaTeX Workshop}. Esta última requiere disponer de la distribución \textit{MiKTeX}. 

\begin{figure}[H]
    \centering
    \begin{minipage}{0.38\textwidth}
        \centering
        \includegraphics[width=0.3\textwidth]{Figures/Logo_VSCode.png}
        \caption{Logo de \textit{Git}}
    \end{minipage} \hspace{2cm}
    \begin{minipage}{0.38\textwidth}
        \centering
        \includegraphics[width=0.8\textwidth]{Figures/Logo_MikTex.png}
        \caption{Logo de \textit{GitHub}}
    \end{minipage}
\end{figure}

Por otro lado, para el diseño de diagramas se ha hecho uso de la aplicación \textit{Draw.io} en su versión de 
escritorio y el programa \textit{InkScape} para el tratamiento de imágenes.

% Insertar imagen VSCode, Latex, MikTex, Draw.io & InkScape

\begin{figure}[H]
    \centering
    \begin{minipage}{0.38\textwidth}
        \centering
        \includegraphics[width=0.4\textwidth]{Figures/Logo_Draw.png}
        \caption{Logo de \textit{Git}}
    \end{minipage} \hspace{2cm}
    \begin{minipage}{0.38\textwidth}
        \centering
        \includegraphics[width=0.4\textwidth]{Figures/Logo_Inkscape.png}
        \caption{Logo de \textit{GitHub}}
    \end{minipage}
\end{figure}

