% ARPEGOS:  Automatized Roleplaying-game Profile Extensible Generator Ontology based System %
% Author : Alejandro Muñoz Del Álamo %
% Copyright 2019 %

% Section 1.3: Glosario de Términos %
\section{Glosario de Términos} \label{Glosario}
\begin{itemize}

    % Añadir WebSemántica

    \item \textbf{\textit{Android}}: \textit{“Sistema operativo que se emplea en dispositivos móviles, por lo general con pantalla táctil. 
    De este modo, es posible encontrar tabletas (tablets), teléfonos móviles (celulares) y relojes equipados con Android, aunque 
    el software también se usa en automóviles, televisores y otras máquinas. Creado por Android Inc., una compañía adquirida por 
    Google en 2005, Android se basa en Linux, un programa libre que, a su vez, está basado en Unix.”} \autocite*{JulianPerezPortoyMariaMerino2015}
    
    \item \textbf{\textit{C\#}}: \textit{“Lenguaje de programación simple, moderno, orientado a objetos y fuertemente tipado. 
    C\# tiene sus raíces en la familia de lenguajes C [\dots] C\# está estandarizado por \textbf{ECMA} International con el 
    estándar \textbf{ECMA-334} y por \textbf{ISO/IEC} con el estándar \textbf{ISO/IEC 23270}.”} \autocite*{AndersHejlsbergScottWiltamuth2003}

    \item \textbf{\textit{Git}}: \textit{“Herramienta de control de versiones particularmente potente, flexible y bajos costes generales 
    que hace del desarrollo colaborativo un placer. Git fue inventado por \textbf{Linus Torvalds} para dar soporte al desarrollo 
    del kernel \textbf{Linux}, pero desde entonces ha probado ser valioso a un amplio rango de proyectos”} \autocite*{JonLoeliger2012}

    \item \textbf{\textit{GitHub}}: \textit{“Servidor de respositorios de código basado en el sistema de control de versiones 
    \textbf{Git}”} \autocite*{Dabbish2012}

    \item \textbf{\textit{IDE}}: \textit{“Un entorno de desarrollo integrado o \textbf{IDE} es una aplicación visual que sirve para 
    la construcción de aplicaciones a partir de componentes.”} \autocite*{LozanoPerez2000} 
        
    \item \textbf{\textit{Metodología de desarrollo}}: \textit{“Una metodología es una colección de procedimientos, técnicas, 
    herramientas y documentos auxiliares que ayudan a los desarrolladores de software en sus esfuerzos por implementar nuevos 
    sistemas de información. Una metodología está formada por fases, cada una de las cuales se puede dividir en sub-fases, 
    que guiarán a los desarrolladores de sistemas a elegir las técnicas más apropiadas en cada momento del proyecto y también 
    a planificarlo, gestionarlo, controlarlo y evaluarlo.”} \autocite*{AmayaBalaguera2015}

    \item \textbf{\textit{Metodología Ágil}}: \textit{“Metodologías que se derivan de la lista de los principios que se encuentran 
    en el \textbf{Manifiesto Ágil}, y están basadas en un desarrollo iterativo que se centra que se centra más en capturar mejor 
    los requisitos cambiantes y la gestión de los riesgos, rompiendo el proyecto en iteraciones de diferente longitud”} \autocite*{AmayaBalaguera2015}
    
    \item \textbf{\textit{Modelo}}: \textit{“Las clases de modelo son clases no 
    visuales que encapsulan los datos de la aplicación. Por lo tanto, se 
    puede considerar que el modelo representa el modelo de dominio de la 
    aplicación, que normalmente incluye un modelo de datos junto con la 
    lógica de validación y negocios.”} \autocite*{MicrosoftMVVM}
    % https://docs.microsoft.com/es-es/xamarin/xamarin-forms/enterprise-application-patterns/mvvm

    \item \textbf{\textit{Modelo de Vista}}: \textit{“El modelo de vista implementa las 
    propiedades y los comandos a los que la vista puede enlazarse y notifica 
    a la vista de cualquier cambio de estado a través de los eventos de 
    notificación de cambios. Las propiedades y los comandos que proporciona 
    el modelo de vista definen la funcionalidad que ofrece la interfaz de 
    usuario, pero la vista determina cómo se mostrará esa funcionalidad.”} \autocite*{MicrosoftMVVM}
    % https://docs.microsoft.com/es-es/xamarin/xamarin-forms/enterprise-application-patterns/mvvm
    
    \item \textbf{\textit{MVVM}}: \textit{“Patrón de arquitectura de software que
    ayuda a separar la lógica de negocios y presentación de una aplicación 
    de su interfaz de usuario. ”} \autocite*{MicrosoftMVVM}
    % https://docs.microsoft.com/es-es/xamarin/xamarin-forms/enterprise-application-patterns/mvvm

    \item \textbf{\textit{Ontología}}: \textit{“Las ontologías computacionales tienen como objetivo 
    modelar la estructura de un sistema, por ejemplp, las entidades y relaciones relevantes que emergen 
    de la observación, que son útiles a nuestros propósitos.[...] Gruber \autocite*{Gruber1995} describe la noción de una ontología 
    como la especificación explícita de una conceptualización compartida”} \autocite*{Guarino2009}

    \item \textbf{\textit{OWL}}: \textit{“El lenguaje de ontología web \textbf{OWL} es un lenguaje de marcado semántico para publicar
    y compartir ontologías en la World Wide Web. OWL está desarrollado como una extensión de vocabulario de \textbf{RDF} y está 
    derivado del lenguaje de ontología web DAML+OIL.”} \autocite*{Bechhofer2004}

    \item \textbf{\textit{Protégé}}: \textit{“El sistema Protégé es un entorno para el desarrollo de sistemas basados en 
    conocimiento que ha evolucionado durante más de una decada. Protégé comenzó comouna pequeña aplicación diseñada para un 
    dominio médico (planificación de terapia basada en protocolos), pero ha evolucionado en un conjunto de herraminetas con un 
    propósito mucho más general [...] El objetivo inicial de Protégé era reducir el cuello de botella de adquisición de conocimiento
    minimizando el rol del ingeniero de conocimiento construyendo bases de conocimiento.”} \autocite*{Gennari2003}

    \item \textbf{\textit{RDF}}: \textit{“El \textbf{F}ramework de \textbf{D}esripción de \textbf{R}ecursos o RDF es una fundación para 
    procesar metadatos; provee interoperabilidad entre aplicaciones que intercambian información comprensible para máquinas en la Web.
    RDF enfatinza las facilidades para permitir el procesamiento automatizado de recursos web. ”} \autocite*{Lassila1999}

    \item \textbf{\textit{RDFS}}: \textit{“RDF y su esquema de extension, RDF Schema Specification (RDFS) forman las dos capas más 
    bajas de la \textbf{Web Semántica}[\dots] RDFS provee un mecanismo estándar para declarar clases y propiedades (globales) así como 
    para definir relaciones entre clases y propiedades, usando sintaxis de RDF.”} \autocite*{Kaoudi2008}

    \item \textbf{\textit{RDFSharp}}: \textit{“RDFSharp es un framework ligero de \textbf{C\#} que puede ser utilizado para 
    realizar aplicaciones, servicios y sitios web capaces de modelar, almacenar y consultar datos \textbf{RDF/SPARQL}.”} 
    \autocite*{DeSalvo41}

    \item \textbf{\textit{Scrum}}: \textit{“Marco de trabajo para la gestión y 
    desarrollo del software basada en un proceso iterativo e incremental 
    utilizado comúnmente en entornos basados en el desarrollo ágil del
    software.”} \autocite*{AlonsoAlvarezGarciaRafaeldelasHerasdelDedo2012}

    \item \textbf{\textit{SPARQL}}: \textit{“SPARQL es un lenguaje de consulta desarrollado pincipalmente para consultar
    grafos RDF.”} \autocite*{Sirin2007}

    \item \textbf{\textit{Sprint}}: \textit{“Período en el cual se lleva el desarrollo de una tarea.”} \autocite*{AlonsoAlvarezGarciaRafaeldelasHerasdelDedo2012}

    \item \textbf{\textit{UML}}: \textit{“El Lenguaje Unificado de Modelado (UML) es, tal como su nombre lo indica, un lenguaje de
    modelado y no un método o un proceso. El UML está compuesto por una notación muy específica y por las reglas semánticas relacionadas 
    para la construcción de sistemas de software. El UML en sí mismo no prescribe ni aconseja cómo usar esta notación en el proceso 
    de desarrollo o como parte de una metodología de diseño orientada a objetos. ”} \autocite*{Sparks2008}

    \item \textbf{\textit{Vista}}: \textit{“La vista es responsable de definir 
    la estructura, el diseño y la apariencia de lo que el usuario ve 
    en la pantalla. Idealmente, cada vista se define en XAML, con un 
    código subyacente limitado que no contiene la lógica de negocios. 
    Sin embargo, en algunos casos, el código subyacente podría contener 
    lógica de la interfaz de usuario que implementa el comportamiento 
    visual que es difícil de expresar en XAML, como animaciones.”}  \autocite*{MicrosoftMVVM}
    % https://docs.microsoft.com/es-es/xamarin/xamarin-forms/enterprise-application-patterns/mvvm

    \item \textbf{\textit{Visual Studio}}: \textit{“Microsoft Visual Studio es un entorno de desarrollo integrado 
    (IDE, por sus siglas en inglés) para Windows, Linux y macOS [\dots] Visual Studio permite a los desarrolladores 
    crear sitios y aplicaciones web, así como servicios web en cualquier entorno compatible con la plataforma .NET 
    (a partir de la versión .NET 2002). Así, se pueden crear aplicaciones que se comuniquen entre estaciones de trabajo, 
    páginas web, dispositivos móviles, dispositivos embebidos y videoconsolas, entre otros.”} \autocite*{VisualStudio}

    \item \textbf{\textit{Xamarin}}:\textit{“Xamarin es una plataforma de código abierto para compilar aplicaciones modernas 
    y con mejor rendimiento para iOS, Android y Windows con .NET. Xamarin es una capa de abstracción que administra la 
    comunicación de código compartido con el código de plataforma subyacente. Xamarin se ejecuta en un entorno administrado 
    que proporciona ventajas como la asignación de memoria y la recolección de elementos no utilizados.”} \autocite*{Xamarin}
    
    \item \textbf{\textit{Xamarin.Forms}}:\textit{“Xamarin.Forms es un marco de interfaz de usuario de código abierto. 
    Xamarin.Forms permite a los desarrolladores compilar aplicaciones en Xamarin.Android, Xamarin.iOS y Windows desde 
    un único código base compartido. Xamarin.Forms permite a los desarrolladores crear interfaces de usuario en XAML 
    con código subyacente en C\#. Estas interfaces se representan como controles nativos con mejor rendimiento en cada plataforma.”}
    \autocite*{XamForms}
    
    \item \textbf{\textit{XML}}:\textit{“XML is a markup language for documents containing structured information. 
    Structured information contains both content (words, pictures, etc.) and some indication of what role that content 
    plays (for example, content in a section heading has a different meaning from content in a footnote, which means something
    different than content in a figure caption or content in a database table, etc.). Almost all documents have some structure.
    A markup language is a mechanism to identify structures in a document. The XML specification defines a standard way to add 
    markup to documents.”} \autocite*{Walsh}

    \item \textbf{\textit{XAML}}: \textit{“XAML significa \textbf{L}enguaje de \textbf{M}arcado de \textbf{A}plicaciones 
    \textbf{E}xtensible. Es el nuevo lenguaje declarativo de Microsoft para definir aplicaciones con interfaces de usuario.
    XAML provee una sintexis accesible, extensible y localizable para definir interfaces de usuario separadas de la lógica de 
    la aplicación. smiliar a la técnica orientada a objeto para desarrollar aplicaciones de múltiples niveles con la arquitectura 
    Modelo-Vista-Controlador.”} \autocite*{MacVittieA.2006}
    
    \item \textbf{\textit{Web Semántica}}: \textit{“Extensión de la actual 
    web en la que a la información disponible se le otorga un significado 
    bien definido que permita a los ordenadores y las personas trabajar en 
    cooperación. Se basa en la idea de tener datos en la web definidos y 
    vinculados de modo que puedan usarse para un descubrimiento, 
    automatización y reutilización entre varias aplicaciones.”} \autocite*{Hendler2002}

\end{itemize}
