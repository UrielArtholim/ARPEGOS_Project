% ARPEGOS:  Automatized Roleplaying-game Profile Extensible Generator Ontology based System %
% Author : Alejandro Muñoz Del Álamo %
% Copyright 2019 %

% Section 1.1: Motivación %

\section{Motivación} \label{Motivacion}
Los juegos de rol, a los que llamaremos \textbf{\textit{RPG}} (\textit{\textbf{R}ole-\textbf{P}laying \textbf{G}ame}) a 
partir de ahora, son juegos con una meta particular, que es \textit{“El objetivo de ambos tipos de juegos, computerizados y 
de otro tipo, es experimentar una serie de aventuras en un mundo imaginario, a través de un personaje avatar o un pequeño grupo de 
personajes cuyas habilidades y poderes crecen conforme el tiempo pasa.”} \autocite*{Adams2010}. Se caracterizan por 
disponer de una extensa cantidad de reglas que permiten a los jugadores conocer las posibilidades de las que disponen para 
interactuar a través de sus personajes.
\medskip

La historia de los juegos de rol modernos comienzan su historia con \textit{\textbf{D}ungeons \textbf{\&} \textbf{D}ragons}, 
juego publicado en 1974 \autocite*{HistoriaRPG}. En pocos años, y debido al éxito del juego, comienzan a aparecer otros juegos de rol de fantasía. 
Algunos copiaban su estilo, como \textit{Chivalry \& Sorcery} o \textit{The Arduin Grimoire}, mientras que otros no se 
relacionaban con el universo de \textit{D\&D}, como \textit{RuneQuest} o \textit{Traveller}. En los años 80, surgieron 
algunos juegos que hoy en día son clásicos, como \textit{La llamada de Cthulhu}, \textit{Paranoia}, \textit{GURPS} o \textit{Ars Magica}.
No se publicaron juegos de rol en lengua española hasta la década de 1990, siendo el primero \textit{Aquelarre}.\medskip

Hoy en día los juegos de rol están extendidos por todo el mundo, y se utilizan de diversas maneras. Estos son algunos ejemplos:
\begin{itemize}
    \item Herramienta educativa: Algunos colegios, como el \textit{Østerskov Efterskole} en Dinamarca, están utilizando juegos de 
    rol de acción en vivo (\textit{LARP}) para enseñar en clase \autocite*{LARPSchool}.
    \item Herramienta de apoyo en el desarrollo de habilidades personales, como la empatía, la tolerancia, la socialización, el trabajo
    en equipo, la creatividad, etc.
    \item Método de entretenimiento para juntarse con amigos y pasar un rato agradable.
\end{itemize}


Uno de los aspectos más importantes de los \textit{RPG} son los personajes, ya que estos son la manifestación de los jugadores 
dentro del universo del juego, que a partir de ahora llamaremos \textbf{personajes jugadores} o \textbf{\textit{PJs}}. 
Ramos y Sueiro~\autocite*{Ramos-Villagrasa2010} plantean que, a diferencia del teatro, donde la elección de un personaje puede depender de las características 
físicas del intérprete, en los juegos de rol cualquier persona puede interpretar cualquier personaje, siempre que esté dentro de las 
posibilidades que ofrezca el juego en que se esté jugando. Los personajes deben definirse dentro de los limites establecidos por 
las reglas del juego, y en una historia como si se tratara de su biografía, explicando cómo ha sido la vida del personaje en 
la ambientación hasta el momento de comenzar la historia. A este proceso de definir un personaje 
para un jugador se conoce como \textbf{creación de personaje}. \medskip

Hay un amplio rango de posibilidades cuando tratamos la creación de \textit{PJs} en diferentes mundos. En algunos juegos, el jugador 
sólo puede seleccionar algunas características predefinidas para el personaje, mientras que en otros juegos el usuario puede cambiar cada
elemento del avatar \autocite*{Isaksson2012}. Esto hace que la creación de personajes sea un proceso arduo y complejo, pues es necesario 
tener amplios conocimientos del juego para conocer todas las opciones de personalización disponibles para el jugador, y sus correspondientes 
características. \medskip

Egri \autocite*{Egri1960} afirma en su libro \textit{The Art of Dramatic Writing: Its Basis in the Creative Interpretation 
of Human Motives} que mientras todo objeto tiene tres dimensiones: profundidad, altura y anchura, los seres humanos tienen 
tres dimensiones adicionales: \textbf{fisiología}, \textbf{sociología} y \textbf{psicología}, y que sin el conocimiento de estas dimensiones, no se puede
valorar a un ser humano. La primera dimensión, la \textbf{fisiología}, da color al punto de vista humano y le influye infinitamente, 
facilitando a una persona ser tolerante, desafiante, humilde o arrogante. La \textbf{sociología}, hace referencia a que no es posible 
realizar un análisis exacto de las diferencias entre una persona y el vecino de la puerta de al lado si no se tiene suficiente 
información de las circunstancias de ambas personas. La tercera dimensión, la \textbf{psicología}, es el producto de las otras dos, cuya 
influencia da vida a la ambición, la frustración, el temperamento, las actitudes y los complejos.\medskip

En base a esto, Lankoski, Heliö y Ekman \autocite*{Lankoski2003} presentan una tabla, llamada 
\textit{Estructura ósea de una persona tridimensional}, que se muestra a continuación, definiendo los aspectos de 
esta estructura definida por Egri, modificando y añadiendo algunos aspectos de la misma.\medskip

\begin{table}[H]
    \centering
    \begin{tabular}{|c|c|c|}
        \hline
        \thead{\textit{\textbf{Fisiología}}} & \thead{\textit{\textbf{Sociología}}} & \thead{\textit{\textbf{Psicología}}} \\
        \hline
        \hline 
        \makecell{Sexo} & \makecell{Clase} & \makecell{Estándares morales y \\ vida sexual} \\
        \makecell{Edad} & \makecell{Ocupación} & \makecell{Metas y ambiciones} \\
        \makecell{Altura y anchura} & \makecell{Educación} & \makecell{Frustraciones y decepciones} \\
        \makecell{Color de pelo, ojos y piel} & \makecell{Vida familiar} & \makecell{Temperamento} \\
        \makecell{Postura} & \makecell{Religión} & \makecell{Actitud frente a la vida} \\
        \makecell{Apariencia y \\ rasgos distintivos} & \makecell{Raza y nacionalidad} & \makecell{Complejos y obsesiones} \\
        \makecell{Defectos} & \makecell{Posición social} & \makecell{Imaginación, juicios, \\ sabiduría, gustos y \\ estabilidad} \\
        \makecell{Rasgos hereditarios} & \makecell{Afiliaciones políticas} & \makecell{Extroversión, \\ introversión  y ambiversión} \\
        \makecell{Físico} & \makecell{Entretenimientos y \\ aficiones} & \makecell{Inteligencia} \\
        \hline
    \end{tabular}
    \caption{\textit{Lankoski, Heliö y Ekman}: Estructura de una persona tridimensional. \autocite*{Lankoski2003}}
\end{table}
% Hablar del proceso de creación %

En lo referido a juegos de rol, \textit{Tychsen}, \textit{Hitchens} y \textit{Brolund} sostienen que 
\textit{“el diseño de personajes está dividido en cuatro componentes: \textbf{personalidad}, \textbf{integración}, \textbf{apariencia}
y \textbf{reglas}, que tienen su propio conjunto de restricciones para asegurar consitencia metodológica”.} \autocite*{Tychsen2008}

\begin{itemize}
    \item Son las características basadas en las reglas del juego, tales como 
    \textit{atributos}, \textit{aptitudes}, \textit{habilidades} y \textit{clase}. Los \textit{RPG} proveen estas características 
    en plantillas conocidas como \textit{hojas de personaje}, que detallan todos los componentes basados en reglas del personaje, 
    referentes al juego al que pertenezca el \textit{PJ} en cuestión.
    
    \item \textbf{\textit{Integración}}: Los componentes de integración son aquellos que explican las 
    circunstancias del personaje. Algunos de estos componentes son:
    
    \begin{itemize}
        \item \textit{Ubicación}: ¿Dónde está el personaje y por qué?
        \item \textit{Trasfondo}: ¿Cual es la historia detrás del personaje y qué eventos lo llevan al 
        punto de inicio del juego?
        \item \textit{Contactos}: ¿Que contactos y relaciones tiene el personaje con otros personajes no controlados por 
        jugadores (conocidos como \textit{personajes no jugadores} o \textbf{\textit{PNJs}})?
        \item \textit{Conexiones}: ¿Cuál es la relación entre el personaje y los demás \textit{PJs}?
    \end{itemize}

    \item \textbf{\textit{Apariencia}}: El personaje requiere una definición de su apariencia, para lo que son necesarios 
    algunos rasgos de importancia:

    \begin{itemize}    
        \item \textit{Representación}: ¿Cómo luce el personaje?
    
        \item \textit{Comportamiento físico}: ¿Cómo se comporta el personaje? ¿Cómo puede su interacción con el mundo 
        modificar su personalidad o comportamiento?
        
        \item \textit{Interación}:¿Cómo interactua el personaje con el entorno y como interacciona con otros personajes?
    \end{itemize}   

    \item \textbf{\textit{Personalidad}}: Este componente incluye descripciones de la psique del personaje (emociones, 
    comportamiento) y metas, con algunos rasgos de personalidad muy definidos enfáticamente (normalmente más sutiles y 
    que dan lugar a la interpretación personal del jugador).
\end{itemize}

De todos los apartados previos, hay uno que destaca por estar plenamente orientado a la información del juego, que es el apartado 
de \textit{Reglas}, ya que es el único que tiene información objetiva del personaje, y que sigue una estructura, dada por el 
\textit{RPG} al que vaya a pertenecer el personaje, que posibilita establecer un conjunto de elementos estructurado, que unido a la
normativa del juego, permita establecer las características técnicas de cualquier personaje. \medskip

En relación con lo comentado previamente, y de acuerdo con el autor del blog \textit{Ars Rolica}~\autocite*{ArsRolica}, 
una de las novedades que se están popularizando más es la de los \textbf{generadores de personajes}, que son herramientas que permiten 
crear personajes de juegos sin necesidad de invertir grandes cantidades de tiempo, evitando posibles errores y deslices. 
Estos generadores pueden ser bastante útiles tanto para jugadores novatos que no conocen el sistema del juego, como para 
jugadores más experimentados que no puedan utilizar tiempo para documentarse completamente antes de crear su álter ego. \medskip

El objetivo de este proyecto consiste en diseñar e implementar un generador de personajes de juegos de rol de mesa, para lo que se 
perseguirán los siguientes subobjetivos:
\begin{itemize}
    \item Realizar un estudio del arte de los \textit{generadores de personaje} existentes, que permita conocer el estado actual de las herramientas
    que se encuentran en uso en la actualidad.

    \item Desarrollar un sistema que permita trabajar con diferentes \textit{RPGs}, sin tener que salir de la aplicación, siempre y 
    cuando se disponga de los ficheros que contengan la información de dichos juegos.

    \item Diseñar una estructura genérica para el sistema de información, de forma que permita a la aplicación poder adaptarse de la forma más completa posible a 
    cada juego, permitiendo profundizar de la misma manera en juegos de alta complejidad que en juegos sencillos.

    \item Crear una interfaz de usuario intuitiva, que facilite al usuario la interacción con la aplicación y ésta resulte cómoda y 
    agradable de utilizar.

\end{itemize}