% ARPEGOS:  Automatized Roleplaying-game Profile Extensible Generator Ontology based System %
% Author : Alejandro Muñoz Del Álamo %
% Copyright 2019 %

% Section 2.4: Riesgos %
\section{Evaluación y gestión de riesgos}
\subsection{¿Qué es un riesgo?}
Se considera \textbf{riesgo} en un proyecto de implantación de software cualquier 
eventualidad que pueda suponer una desviación del plan previsto y que posibilite el 
fracaso del proyecto. 

\subsection{Gestión de riesgos}
La gestión de riesgos permite al equipo de desarrollo definir de forma lógica una serie 
de actividades para analizar los riesgos que se puedan presentar a lo largo del ciclo
de vida del proyecto, calcular su exposición y priorizarlos, de manera que se puedan 
establecer estrategias de control, resolución y supervisión de los mismos.

\subsection{Identificación de riesgos}
En primer lugar, se procederá a realizar una lista que identificará los posibles riesgos
que puedan surgir a lo largo del proyecto:
\begin{itemize}

    \item \textbf{Riesgos de planificación}: La planificación mal realizada puede retrasar 
    enormemente el proyecto, pudiendo resultar en el fracaso del mismo. Posibles motivos para 
    esto son una mala estimación de los tiempos de desarrollo, imposición de plazos por parte del 
    cliente o una planificación optimista.

    \item \textbf{Riesgos de organización y gestión del proyecto}: Es posible que una parte del personal 
    abandone el proyecto, lo que puede derivarse en el fracaso del proyecto si no se encuentra un sustituto 
    adecuado rápidamente.

    \item \textbf{Riesgos de infraestructura hardware y software}: Es posible que surjan problemas con las 
    herramientas, tales como incompatibilidad entre herramientas o incompatibilidad entre herramientas y dispositivos.

    \item \textbf{Riesgos de requisitos}: Es posible que los requisitos generen contratiempos, por ser añadidos constantemente 
    o por modificar drásticamente lo previamente desarrollado.

    \item \textbf{Riesgos de diseño e implementación}: Es posible la falta de un diseño adecuado provoque obstáculos, ya sea 
    por no realizar el diseño y pasar directamente a la implementación, porque el diseño esté demasiado simplificado para 
    la complejidad del proyecto, o por tratar de implementar funciones no soportadas por las herramientas de trabajo.

\end{itemize}

\subsection{Reducción de riesgos}
En esta sección se van a describir las medidas que el equipo ha considerado oportunas para minimizar la aparición de 
los riesgos previamente indicados.

\begin{itemize}

    \item \textbf{Riesgos de planificación}: Se realizará una planificación objetiva, considerando que 
    es susceptible a tener problemas, que en el caso de darse, se estudiarán para modificar la planificación 
    existente de manera que se puedan sortear minimizando su efecto en el desarrollo del proyecto.

    \item \textbf{Riesgos de organización y gestión del proyecto}: Es posible que una parte del personal 
    abandone el proyecto, lo que puede derivarse en el fracaso del proyecto si no se encuentra un sustituto 
    adecuado rápidamente.

    \item \textbf{Riesgos de infraestructura hardware y software}: Se hará un estudio previo de las herramientas para 
    comprobar que es posible trabajar con ellas sin problema alguno. También se realizarán copias de seguridad periódicamente, 
    de manera que de darse un riesgo imprevisible como que el dispositivo de almacenamiento sufra un error y la información 
    sea irrecuperable, haya alguna manera de obtener una versión anterior del proyecto, lo más actualizada posible, desde la que 
    se pueda partir. 

    \item \textbf{Riesgos de requisitos}: Se realizará una recolección exhaustiva de requisitos del proyecto al comienzo, para 
    evitar en la medida de lo posible la aparición de requisitos extra durante el desarrollo del proyecto. Todos los requisitos 
    quedarán documentados.

    \item \textbf{Riesgos de diseño e implementación}: En cada etapa del desarrollo, se comprobará de forma periódica si el diseño 
    puede obviar alguno de los requisitos establecidos.

\end{itemize}
