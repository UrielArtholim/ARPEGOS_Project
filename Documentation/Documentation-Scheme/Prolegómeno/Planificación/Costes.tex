% ARPEGOS:  Automatized Roleplaying-game Profile Extensible Generator Ontology based System %
% Author : Alejandro Muñoz Del Álamo %
% Copyright 2019 %

% Section 2.4: Costes %

\section{Costes}
\subsection{Costes humanos}
Para calcular los costes humanos del proyecto, es necesario conocer el sueldo del equipo de desarrollo.
Para tomar una referencia lo más realista posible, se ha desglosado el trabajo realizado en roles según 
las funciones que se han desempeñado y el tiempo que se ha dedicado a esas funciones. Además, se ha buscado 
el sueldo medio anual de cada rol y se han calculado el coste medio mensual con 14 pagas (12 pagas mensuales 
y 2 pagas extra), y el coste medio diario. \medskip

\begin{table}[H]
    \centering
    \STautoround{0}
    \STsetdecimalsep{,}
    \begin{spreadtab}{{tabular}{|c|c|c|c|}}
        \hline
        \multirow{2}{*}{@\textbf{Rol}} & @\textbf{Salario medio} & @\textbf{Salario medio} & @\textbf{Salario medio}\\
        & @\textbf{anual} & @\textbf{mensual} & @\textbf{diario} \\\hline \hline 
        @Business Analyst & \EUR{:={32417}} & \EUR{:={b3/14}} & \EUR{:={c3/30}} \\\hline %\EUR{:={}}\\\hline
        @Solution Architect & \EUR{:={31700}} & \EUR{:={b4/14}} & \EUR{:={c4/30}} \\\hline
        @UX Designer & \EUR{:={26600}} & \EUR{:={b5/14}} & \EUR{:={c5/30}} \\\hline
        @Developer & \EUR{:={27138}} & \EUR{:={b6/14}} & \EUR{:={c6/30}} \\\hline
        @Tester & \EUR{:={26324}} & \EUR{:={b7/14}} & \EUR{:={c7/30}} \\\hline
        @Documentalist & \EUR{:={31165}} & \EUR{:={b8/14}} & \EUR{:={c8/30}} \\\hline
    \end{spreadtab}\par\smallskip
    \caption{Tabla de sueldos basados en roles}
    \label{Tabla_sueldos}
\end{table}

Una vez que ya disponemos del salario diario según los roles, procedemos a calcular los costes humanos en función 
del tiempo empleado en las funciones de cada rol. \medskip

\begin{table}[H]
    \centering
    \STautoround{0}
    \STsetdecimalsep{,}
    \begin{spreadtab}{{tabular}{|c|c|c|c|}}
        \hline
        @\textbf{Rol} & @\textbf{Salario diario} & @\textbf{Tiempo (días)} & @\textbf{Coste/día}\\\hline \hline 
        @Business Analyst & {:={3}} & \EUR{:={77}} & \EUR{:={b2*c2}} \\\hline 
        @Solution Architect & {:={158}} & \EUR{:={75}} & \EUR{:={b3*c3}} \\\hline
        @UX Designer & {:={5}} & \EUR{:={63}} & \EUR{:={b4*c4}} \\\hline
        @Developer & {:={34}} & \EUR{:={65}} & \EUR{:={b5*c5}} \\\hline
        @Tester & {:={4}} & \EUR{:={63}} & \EUR{:={b6*c6}} \\\hline
        @Documentalist & {:={9}} & \EUR{:={75}} & \EUR{:={b7*c7}} \\\hline \hline
        @\textbf{Total} & \textbf{{:={sum(b2:b7)}}} &  & \textbf{\EUR{:={sum(d2:d7)}}} \\ \hline
    \end{spreadtab}\par\smallskip
    \caption{Costes humanos del proyecto}
    \label{Costes humanos}
\end{table}

\newpage
\subsection{Costes materiales}
Para el cálculo de los costes materiales se han tenido en cuenta los siguientes elementos:\medskip

\begin{itemize}
    \item \textit{Hardware}: Elementos físicos que se han utilizado para el desarrollo del proyecto
    \item \textit{Software}: Programas de los que se han hecho uso para la elaboración de la aplicación
    \item \textit{Varios}: Costes varios de elementos necesarios para poder realizar el trabajo 
    correctamente, tales como el gasto de luz o el coste de internet.
\end{itemize}

\subsubsection{Hardware}

Para calcular el coste del hardware, hemos contemplado el valor que queda por amortizar 
del ordenador que se ha utilizado para trabajar. \medskip

\begin{table}[H]
    \centering
    \STautoround{0}
    \STsetdecimalsep{,}
    \begin{spreadtab}{{tabular}{|c|c|c|c|c|c|}}
        \hline
        @\textbf{Elemento} & @\textbf{Valor} & @\textbf{Vida útil} & @\textbf{Tiempo de vida} & @\textbf{Valor/año} \\\hline \hline 
        @Ordenador & \EUR{:={1726}} & {:={5}} & {:={4}} & \EUR{:={(b2/c2)*(c2-d2)}} \\\hline \hline
        @\textbf{Total} & & & & \textbf{\EUR{:={e2}}} \\ \hline
    \end{spreadtab}\par\smallskip
    \caption{Costes de hardware}
    \label{Coste_hardware}
\end{table}

\subsubsection{Software}
Como todo el software que se ha utilizado es de uso gratuito, el coste en este apartado es de \EUR{0}.\medskip

\begin{table}[H]
    \centering
    \STautoround{0}
    \STsetdecimalsep{,}
    \begin{spreadtab}{{tabular}{|c|c|}}
        \hline
        @\textbf{Software} & @\textbf{Coste/día}\\\hline \hline 
        @Visual Studio 2019 Community & \EUR{:={0}}  \\\hline 
        @Xamarin Forms & \EUR{:={0}}  \\\hline
        @Protégé & \EUR{:={0}}  \\\hline
        @Visual Studio Code & \EUR{:={0}}  \\\hline
        @\LaTeX Workshop & \EUR{:={0}}  \\\hline
        @MiKTeX & \EUR{:={0}} \\\hline \hline
        @\textbf{Total} & \textbf{\EUR{:={sum(b2:b7)}}} \\ \hline
    \end{spreadtab}\par\smallskip
    \caption{Costes de software}
    \label{Coste_software}
\end{table}

\subsubsection{Varios}
En este apartado se han considerado los gastos de electricidad y conexión a Internet, que han 
sido indispensables para llevar a cabo el trabajo.\medskip

\begin{table}[H]
    \centering
    \STautoround{0}
    \STsetdecimalsep{,}
    \begin{spreadtab}{{tabular}{|c|c|c|c|}}
        \hline
        @\textbf{Software} & @\textbf{Coste/mes} & @\textbf{Coste/día} & @\textbf{Coste total} \\\hline \hline 
        @Luz & \EUR{:={57}} & \EUR{:={b2/30}} & \EUR{:={c2*215}} \\\hline 
        @Internet & \EUR{:={50}} & \EUR{:={b3/30}} & \EUR{:={c3*215}} \\\hline \hline
        @\textbf{Total} & & & \textbf{\EUR{:={sum(d2:d3)}}} \\ \hline
    \end{spreadtab}\par\smallskip
    \caption{Costes varios}
    \label{Coste_varios}
\end{table}

Tras calcular los diferentes costes materiales por separado, solo falta sumar sus valores para obtener el coste material total.
\medskip

\begin{table}[H]
    \centering
    \STautoround{0}
    \STsetdecimalsep{,}
    \begin{spreadtab}{{tabular}{|c|c|c|c|}}
        \hline
        @\textbf{Elementos} & @\textbf{Coste}  \\\hline \hline 
        @Hardware & \EUR{:={345}} \\\hline 
        @Software & \EUR{:={0}} \\\hline 
        @Varios & \EUR{:={860}}\\\hline \hline
        @\textbf{Total} & \textbf{\EUR{:={sum(b2:b4)}}} \\ \hline
    \end{spreadtab}\par\smallskip
    \caption{Costes varios}
    \label{Coste_material}
\end{table}

\subsection{Costes totales}
Calculados los costes humanos y materiales, podemos obtener el coste total del proyecto realizando la suma 
de estos. \medskip

\begin{table}[H]
    \centering
    \STautoround{0}
    \STsetdecimalsep{,}
    \begin{spreadtab}{{tabular}{|c|c|c|c|}}
        \hline
        @\textbf{Elementos} & @\textbf{Coste}  \\\hline \hline 
        @Costes humanos & \EUR{:={15533}} \\\hline 
        @Costes materiales & \EUR{:={1205}} \\\hline 
        @\textbf{Total} & \textbf{\EUR{:={sum(b2:b3)}}} \\ \hline
    \end{spreadtab}\par\smallskip
    \caption{Coste total del proyecto}
    \label{Coste_total}
\end{table}


