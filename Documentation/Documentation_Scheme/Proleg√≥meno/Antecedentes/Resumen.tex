% ARPEGOS:  Automatized Roleplaying-game Profile Extensible Generator Ontology based System %
% Author : Alejandro Muñoz Del Álamo %
% Copyright 2019 %

% Section 2.4: Resumen %
\section{Resumen}
Las aplicaciones actuales permiten generar información útil para los jugadores de rol, tal como referencias rápidas, 
tiradas de dados, resultados de combates e incluso la información de sus personajes. Pero la mayoría de ellas están limitadas
en que su ámbito suele estar reducido a una única versión de un juego específico. Además, no pueden personalizarse para el jugador, 
o requieren que se introduzca la información a mano cada vez que se utilizan, resultando ser menos práctico que el método tradicional.
\medskip

En relación a esto, es de gran interés promover el uso de ontologías en este tipo de aplicaciones, como este proyecto, por ejemplo, 
pues permiten relacionar elementos de diferentes ontologías entre sí, permitiendo reutilizar elementos ya definidos, y ampliar
modelos de datos sin realizar modificaciones de las versiones previas, posibilitando el uso de nuevas versiones de los modelos
sin alterar los ya existentes, facilitando al usuario la posibilidad de escoger la versión que considere más adecuada para cada ocasión.
\medskip

Por otro lado, con respecto al desarrollo de este proyecto se ha decidido optar por la metodología \textit{Scrum} porque 
permite al equipo de desarrollo trabajar con requisitos concretos a corto plazo, priorizando los requisitos más valorados 
por los clientes, y que permite a éstos últimos disponer de resultados al final de cada iteración, de manera que pueden 
tomar las decisiones que consideren oportunas para la siguiente iteración del desarrollo.

