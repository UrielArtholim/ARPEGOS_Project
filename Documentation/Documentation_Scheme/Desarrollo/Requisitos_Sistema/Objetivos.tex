% ARPEGOS:  Automatized Roleplaying-game Profile Extensible Generator Ontology based System %
% Author : Alejandro Muñoz Del Álamo %
% Copyright 2019 %

% Section 3.2: Objetivos del Sistema %
\section{Objetivos del Sistema}
El sistema va a tener tres finalidades claramente diferenciadas:
\begin{itemize}

    \item \textbf{Selección de juego}:La aplicación podrá dar 
    acceso a varios juegos, de manera que se pueda alternar 
    entre éstos, permitiendo utilizar el mismo mecanismo para 
    el catálogo de juegos disponible.

    \item \textbf{Creación/Modificación de personaje}: La 
    aplicación permitirá al usuario seleccionar la información
    necesaria para la creación de un personaje a su gusto 
    mediante un proceso guiado paso a paso. En caso de que el 
    personaje ya exista, permitirá al usuario visualizar y realizar 
    modificaciones a la información mostrada.

    \item \textbf{Automatización del proceso de cálculo 
    de habilidades}: La aplicación facilitará al usuario una 
    interfaz en la que, al indicar la habilidad que desea 
    utilizar, y el resultado de su lanzamiento de dados, se 
    devuelva el resultado total que se debe aplicar en el 
    combate. También podrá calcular el resultado de tiradas 
    enfrentadas.

\end{itemize}

Además, la aplicación deberá hacer muestra de las siguientes 
cualidades: 
\begin{itemize}

    \item \textbf{Generalidad}: La aplicación no estará directamente 
    vinculada a la información específica de un juego, de forma que 
    sea posible procesar diferentes bancos de datos, y por tanto, 
    se pueda utilizar la misma aplicación para varias versiones 
    distintas del mismo juego, o incluso para juegos completamente 
    diferentes.

    \item \textbf{Sencillez}: La aplicación dispondrá de una interfaz 
    simple y agradable, que permita al usuario hacer uso de sus 
    funciones de forma asequible, sea cual fuere la complejidad del 
    juego seleccionado.

\end{itemize}
