% ARPEGOS:  Automatized Roleplaying-game Profile Extensible Generator Ontology based System %
% Author : Alejandro Muñoz Del Álamo %
% Copyright 2019 %

% Section 3.3: Catálogo de Requisitos %
En la presente sección vamos a proceder a realizar un análisis de 
requisitos del sistema, que recoge y describe el conjunto 
de requisitos específicos del sistema que se va a desarrollar.\medskip

En primera instancia, se presentarán los requisitos agrupados en 
conjuntos funcionales del sistema. Posteriormente, se describirán 
los casos de uso en el próximo capítulo. \medskip

Para ello, se hará una diferenciación entre 
\textit{requisitos funcionales}, que son aquellos que detallan 
la funcionalidad del sistema, y \textit{requisitos no funcionales},
que refieren a otros aspectos del software que deben ser satisfechos.

\subsection{Requisitos funcionales}
Un requisito funcional especifica una función concreta del sistema o 
de alguno de sus componentes. A continuación se muestran los requisitos 
funcionales, estructurados según el módulo del sistema al que refieren.
%\begin{itemize}
    %\item \textbf{\textit{OBJ-001}}: \textbf{Almacenamiento de información}.
    %La aplicación debe disponer de una estructura que permita almacenar y 
    %clasificar la información contenida.
    %\begin{itemize}
        
    %    \item El sistema debe disponer de un conjunto de directorios, en el que 
    %    cada directorio hace referencia a un juego diferente.

    %    \item Cada directorio de juego estará a su vez compuesto por dos 
    %    directorios: uno para almacenar los ficheros referentes a la información 
    %    del juego (\textit{gamefiles}), y otro para almacenar los personajes generados con ese juego
    %    (\textit{characters}).

    %    \item El directorio \textit{gamefiles} contendrá un fichero en formato \textbf{OWL} 
    %    por cada versión accesible del juego en cuestión.
        
    %    \item El directorio \textit{characters} sólo contendrá los ficheros de personaje 
    %    generados por el sistema durante el proceso de creación de personaje.
        
    %\end{itemize}

    %\item \textbf{\textit{OBJ-001}}: \textbf{Lógica de la aplicación}.
    El sistema precisa de una lógica estructurada y compleja que permita procesar 
    información de diferentes fuentes, de manera que los procesos del sistema se 
    adecuen a su contenido.
    \begin{itemize}
        
        \item \textbf{\textit{OBJ-001}}: El usuario podrá seleccionar un juego concreto 
        (juego activo) para poder acceder a la información relacionada con el mismo.

        \item \textbf{\textit{OBJ-002}}: El usuario podrá crear un personaje para el juego activo,
        mediante un proceso guiado paso a paso
        
        \item \textbf{\textit{OBJ-003}}: El usuario podrá seleccionar un personaje (personaje activo) 
        de todos los existentes para el juego activo
        
        \item \textbf{\textit{OBJ-004}}: El usuario podrá visualizar la información del 
        personaje seleccionado del juego activo.
        
        \item \textbf{\textit{OBJ-005}}: El usuario podrá eliminar un personaje ya creado 
        del juego activo.
        
        \item \textbf{\textit{OBJ-006}}: El usuario podrá realizar cálculos con los valores 
        de las habilidades del personaje activo.

    \end{itemize}
%\end{itemize}


\subsection{Requisitos no funcionales}
Un requisito no funcional es una propiedad o cualidad que no forma parte de los 
fundamentos del sistema, pero es necesario para que el producto cumpla con su
cometido apropiadamente. \medskip

Para la declaración de requisitos no funcionales, se establecerán como base los 
requisitos indicados en las normas \textit{IEEE Std. 830} e \textit{ISO/IEC 25010 (SQuaRE)}:

\begin{itemize}   
    \item \textbf{Adecuación funcional}: La aplicación debe cumplir con todos los requisitos necesarios, 
    de manera que sea completo y correcto funcionalmente.

    \item \textbf{Seguridad}: El sistema no requiere asegurar la información que procesa, debido a que 
    no contiene información sensible del usuario en ningún momento, ni realiza conexión externa alguna 
    para obtener información.
    
    \item \textbf{Compatibilidad}: La aplicación deberá ser compatible con los ficheros que contienen
    la información de los juegos que formarán parte del sistema.
    
    \item \textbf{Usabilidad}: El sistema debe disponer de una interfaz de usuario intuitiva y fácil de 
    manejar, de manera que pueda ser utilizado por usuarios sin conocimientos técnicos ni avanzados de 
    informática. La curva de aprendizaje deberá ser lo más reducida posible, de manera que personas de 
    cualquier ámbito puedan hacer uso del mismo.

    \item \textbf{Fiabilidad}: La aplicación deberá estar libre de errores que influyan negativamente 
    en su uso normal. Debido a que la aplicación depende de información incluida por terceros, será 
    necesario comprobar que dicha información es compatible con la aplicación.
    
    \item \textbf{Eficiencia}: El sistema debe evitar en la medida de lo posible el uso de información 
    redundante para poder asegurar su funcionamiento cuando se introduzcan juegos de alta complejidad 
    que requieran un elevado uso de recursos.

    \item \textbf{Mantenibilidad}: Este apartado representa la capacidad del producto software para 
    ser modificado efectiva y eficientemente. Esto será posible debido al desarrollo de código 
    limpio y bien documentado, al diseño y la implementación modular del mismo. Se plantea el uso 
    de patrones de arquitectura de software, tales como \textbf{MVVM}.

    \item \textbf{Portabilidad}: El sistema está diseñado para su uso en dispositivos móviles, 
    aunque no se descarta una futura ampliación para introducirlo en otro tipo de dispositivos.
    La implementación está realizada únicamente para sistemas \textit{Android}, ya que no 
    se dispone de las herramientas necesarias para el despliegue en \textit{Mac OS}. 
\end{itemize}
