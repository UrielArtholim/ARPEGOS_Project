% ARPEGOS:  Automatized Roleplaying-game Profile Extensible Generator Ontology based System %
% Author : Alejandro Muñoz Del Álamo %
% Copyright 2019 %

% Section 1.4: Organización del Documento %

\section{Estructuración del documento}
El documento que se presenta está estructurado en una colección de capítulos, 
en los que se describen de manera precisa y detallada todas y cada una de 
las etapas por las que ha pasado el proyecto, desde su inicio hasta su 
conclusión. A continuación se muestra un breve resumen de los contenidos 
de cada capítulo:

\begin{itemize}
    \item \textbf{\emph{Capítulo 1. Introducción}}. El capítulo inicial 
    consiste en una introducción al proyecto, explicando los antecedentes 
    a su desarrollo, así como la motivación para ponerlo en práctica, los 
    objetivos que debe cumplir y un glosario de términos para facilitar 
    la comprensión del presente documento.

    \item \textbf{\emph{Capítulo 2. Antecedentes}}. Tras la introducción,
    se abordan los fundamentos necesarios para simplificar la lectura de 
    este documento. A su vez, se exponen las diferentes tecnologías de las 
    que se han aplicado para la consecución y puesta en marcha de este 
    proyecto.

    \item \textbf{\emph{Capítulo 3. Planificación del proyecto}}. Este 
    capítulo engloba toda la información relevante a aspectos de 
    suma importancia tales como la evaluación de riesgos o la planificación 
    temporal del proyecto.

    \item \textbf{\emph{Capítulo 4-\@8. Análisis, Diseño, Codificación y 
    Pruebas}}. Este conjunto de capítulos profundizan en los diversos 
    aspectos y fases que comprenden el desarrollo de un producto haciendo 
    uso de una \emph{metodología ágil}. Esto posibilita analizar y 
    enriquecer el producto en su desarrollo, mejorando así el resultado 
    final.

    \item \textbf{\emph{Capítulo 9. Manual de Instalación}}. Aquí se 
    ha desarrollado un documento con las instrucciones necesarias 
    para realizar la instalación de la aplicación.

    \item \textbf{\emph{Capítulo 10. Manual de Usuario}}. Se ha redactado 
    un manual de usuario para la aplicación, cuyo objetivo es garantizar un 
    uso eficiente y responsable de la misma por parte de los usuarios. 
    
    \item \textbf{\emph{Capítulo 11. Conclusiones}}. El último capítulo es 
    un breve repaso sobre el desarrollo del proyecto, que incluye una 
    opinión personal y un apartado de posibles mejoras que se podrían 
    realizar al proyecto en un futuro.
\end{itemize}
