% ARPEGOS:  Automatized Roleplaying-game Profile Extensible Generator Ontology based System %
% Author : Alejandro Muñoz Del Álamo %
% Copyright 2019 %

% Section 1.3: Glosario de Términos %
\section{Glosario de Términos}
\begin{itemize}

    % Añadir WebSemántica

    \item \textbf{\textit{Android}}: Sistema operativo que se emplea 
    en dispositivos móviles, por lo general con pantalla táctil. 
    De este modo, es posible encontrar tabletas, teléfonos móviles 
    y relojes equipados con Android, aunque el software también 
    se usa en otros dispositivos. 
    % https://definicion.de/android/ 

    \item \textbf{\textit{Blazegraph}}: Base de datos de grafos de 
    código abierto, escalable y de alto rendimiento basada en 
    estándares. Escrito completamente en \emph{Java}, la plataforma soporta 
    las familias de especificaciones \emph{Blueprint} y \textbf{RDF}/\textbf{SPARQL 1.1}
    incluyendo consultas, actualizaciones, consultas federadas básicas
    y descripción de servicios. 
    % https://wiki.blazegraph.com/wiki/index.php/About_Blazegraph
    
    \item \textbf{\textit{C\#}}: Lenguaje de programación multiparadigma 
    desarrollado y estandarizado por Microsoft como parte de su 
    plataforma \textbf{\dotnet}, que después fue aprobado como un estándar por la
    ECMA (\emph{ECMA-334}) e ISO (\emph{ISO/IEC 23270}). 
    C\# es uno de los lenguajes de programación diseñados para la 
    infraestructura de lenguaje común. Su sintaxis básica deriva de 
    \emph{C/C++} y utiliza el modelo de objetos de la plataforma \\dotnet, 
    similar al de \emph{Java}, aunque incluye mejoras derivadas de otros lenguajes.
    % https://es.wikipedia.org/wiki/C_Sharp


    \item \textbf{\textit{RDFSharp}}: \emph{Framework} de código 
    abierto \textbf{C\#} diseñado para facilitar la creación de 
    aplicaciones \emph{\dotnet} basadas en el modelo \textbf{RDF}, 
    que representa una solución didáctica directa para comenzar 
    a trabajar con conceptos de \emph{Semántica Web}.
    % https://www.w3.org/2001/sw/wiki/RDFSharp

    \item \textbf{\textit{Git}}: Software de control de versiones 
    diseñado por \emph{Linus Torvalds}, pensando en la eficiencia 
    y la confiabilidad del mantenimiento de versiones de aplicaciones 
    cuando éstas tienen un gran número de archivos de código fuente. 
    Su propósito es llevar registro de los cambios en archivos de 
    computadora y coordinar el trabajo que varias personas realizan 
    sobre archivos compartidos.
    % https://es.wikipedia.org/wiki/Git

    \item \textbf{\textit{GitHub}}: GitHub es una plataforma de desarrollo 
    colaborativo de software para alojar proyectos utilizando el sistema de 
    control de versiones \textbf{Git}.
    % https://conociendogithub.readthedocs.io/en/latest/data/introduccion/

    \item \textbf{\textit{IDE}}: \emph{(Integrated Development Environment)} 
    Aplicación con numerosas características que se pueden usar para muchos 
    aspectos del desarrollo de software.
    %  https://docs.microsoft.com/es-es/visualstudio/get-started/visual-studio-ide?view=vs-2019

    \item \textbf{\textit{iOS}}: Sistema operativo móvil de la multinacional
    \emph{Apple Inc}. Originalmente desarrollado para el iPhone (iPhone OS), 
    después se ha usado en dispositivos como el iPod touch y el iPad. 
    No permite su instalación en hardware de terceros.
    % https://es.wikipedia.org/wiki/IOS
    
    \item \textbf{\textit{Metodología Ágil}}: Las metodologías ágiles son 
    métodos de desarrollo de software en los que las necesidades y soluciones 
    evolucionan a través de una colaboración estrecha entre equipos 
    multidisciplinarios. Se caracterizan por enfatizar la comunicación 
    frente a la documentación, por el desarrollo evolutivo y por su 
    flexibilidad.
    % https://es.wikiversity.org/wiki/Metodolog%C3%ADas_%C3%A1giles_de_desarrollo_software
    
    \item \textbf{\textit{Modelo}}: Las clases de modelo son clases no 
    visuales que encapsulan los datos de la aplicación. Por lo tanto, se 
    puede considerar que el modelo representa el modelo de dominio de la 
    aplicación, que normalmente incluye un modelo de datos junto con la 
    lógica de validación y negocios. 
    % https://docs.microsoft.com/es-es/xamarin/xamarin-forms/enterprise-application-patterns/mvvm

    \item \textbf{\textit{Modelo de Vista}}: El modelo de vista implementa las 
    propiedades y los comandos a los que la vista puede enlazarse y notifica 
    a la vista de cualquier cambio de estado a través de los eventos de 
    notificación de cambios. Las propiedades y los comandos que proporciona 
    el modelo de vista definen la funcionalidad que ofrece la interfaz de 
    usuario, pero la vista determina cómo se mostrará esa funcionalidad.
    % https://docs.microsoft.com/es-es/xamarin/xamarin-forms/enterprise-application-patterns/mvvm
    
    \item \textbf{\textit{MVVM}}: Patrón de arquitectura de software que
    ayuda a separar la lógica de negocios y presentación de una aplicación 
    de su interfaz de usuario. 
    % https://docs.microsoft.com/es-es/xamarin/xamarin-forms/enterprise-application-patterns/mvvm

    \item \textbf{\textit{Ontología}}:  Definición formal de tipos, propiedades, 
    y relaciones entre entidades que realmente o fundamentalmente existen 
    para un dominio de discusión en particular. Es una aplicación práctica 
    de la ontología filosófica, con una taxonomía.
    %  https://es.wikipedia.org/wiki/Ontolog%C3%ADa_(inform%C3%A1tica)

    \item \textbf{\textit{OWL}}: \emph{(Ontology Web Language)} es un lenguaje 
    de marcado semántico para publicar y compartir ontologías en la 
    World Wide Web. OWL se desarrolla como una extensión de vocabulario 
    de \textbf{RDF} y es derivado del lenguaje DAML + OIL asi.
    % https://www.w3.org/TR/owl-ref/


    \item \textbf{\textit{Protégé}}: Framework editor de ontologías de 
    código abierto y gratuito para construir sistemas inteligentes.
    % https://protege.stanford.edu/

    \item \textbf{\textit{RDF}}: \emph{(Resource Description Framework)} 
    Modelo estándar para el intercambio de datos en la Web. RDF 
    tiene características que facilitan la fusión de datos incluso si 
    los esquemas subyacentes difieren, y admite específicamente la 
    evolución de los esquemas a lo largo del tiempo sin requerir que 
    se cambien todos los consumidores de datos.
    % https://www.w3.org/RDF/

    \item \textbf{\textit{RDFS}}: \emph{RDF Schema} es una 
    extensión del vocabulario básico de \emph{RDF} que proporciona un 
    vocabulario de modelado de datos para los datos relativos a este modelo.
    % https://www.w3.org/TR/rdf-schema/

    \item \textbf{\textit{Scrum}}: Marco de trabajo para la gestión y 
    desarrollo del software basada en un proceso iterativo e incremental 
    utilizado comúnmente en entornos basados en el desarrollo ágil del
    software.
    % Carmen Lasa Gómez Alonso Alvarez García, Rafael de las Heras del Dedo. 
    % "Métodos Ágiles y Scrum". Anaya Multimedia, 2012.
    
    \item \textbf{\textit{SonarQube}}: Herramienta de revisión automática 
    de código para detectar errores, vulnerabilidades y olores de código 
    en su código. Se puede integrar con su flujo de trabajo existente 
    para permitir la inspección continua de código en todas las ramas 
    de su proyecto y solicitudes de extracción.
    % https://docs.sonarqube.org/latest/

    \item \textbf{\textit{SPARQL}}: \emph{(SPARQL Protocol and RDF Query 
    Language)} es un lenguaje y protocolo de consulta para \emph{RDF}. 
    % https://www.w3.org/TR/rdf-sparql-protocol/

    \item \textbf{\textit{Sprint}}: Período en el cual se lleva el 
    desarrollo de una tarea.
    % Carmen Lasa Gómez Alonso Alvarez García, Rafael de las Heras del Dedo. 
    % "Métodos Ágiles y Scrum". Anaya Multimedia, 2012.

    \item \textbf{\textit{Tarsier}}: Herramienta para la visualización 
    interactiva en 3D de grafos RDF.\@
    % Fabio Viola, Luca Roffia, Francesco Antoniazzi, Alfredo D’Elia, Cristiano Aguzzi and Tullio Salmon Cinotti     
    % "Interactive 3D Exploration of RDF Graphs through Semantic Planes" 17 August 2018

    \item \textbf{\textit{UML}}: \emph{(Unified Modeling Language)} es el 
    lenguaje de modelado de sistemas de software más conocido y utilizado 
    en la actualidad.
    % https://es.wikipedia.org/wiki/Lenguaje_unificado_de_modelado

    \item \textbf{\textit{Vista}}: La vista es responsable de definir 
    la estructura, el diseño y la apariencia de lo que el usuario ve 
    en la pantalla. Idealmente, cada vista se define en XAML, con un 
    código subyacente limitado que no contiene la lógica de negocios. 
    Sin embargo, en algunos casos, el código subyacente podría contener 
    lógica de la interfaz de usuario que implementa el comportamiento 
    visual que es difícil de expresar en XAML, como animaciones.
    % https://docs.microsoft.com/es-es/xamarin/xamarin-forms/enterprise-application-patterns/mvvm

    \item \textbf{\textit{Visual Studio}}: El \emph{IDE} de Visual Studio 
    es un panel de inicio creativo que se puede usar para editar, 
    depurar y compilar código y, después, publicar una aplicación.
    Más allá del editor estándar y el depurador que proporcionan la 
    mayoría de \emph{IDE}, Visual Studio incluye compiladores, herramientas 
    de finalización de código, diseñadores gráficos y muchas más 
    características para facilitar el proceso de desarrollo de software.
    % https://docs.microsoft.com/es-es/visualstudio/get-started/visual-studio-ide?view=vs-2019

    \item \textbf{\textit{W3C}}: \emph{(World Wide Web Consortium)} Consorcio 
    internacional que produce recomendaciones para la \emph{WWW}.
    % https://www.w3c.es/Consorcio/

    \item \textbf{\textit{WWW}}: \emph{(World Wide Web)} Sistema de distribución 
    de información basado en hipertexto o hipermedios enlazados y accesibles 
    a través de Internet. 
    % https://es.wikipedia.org/wiki/World_Wide_Web

    \item \textbf{\textit{Xamarin}}: Xamarin es una plataforma de código 
    abierto para compilar aplicaciones modernas y de rendimiento para iOS, 
    Android y Windows con \dotnet.
    % https://docs.microsoft.com/es-es/xamarin/get-started/what-is-xamarin

    \item \textbf{\textit{Xamarin.Forms}}: Xamarin.Forms es un marco de 
    interfaz de usuario de código abierto. Xamarin.Forms permite a los 
    desarrolladores compilar aplicaciones de Android, iOS y Windows 
    desde un único código base compartido.
    % https://docs.microsoft.com/es-es/xamarin/get-started/what-is-xamarin-forms

    \item \textbf{\textit{XAML}}: \emph{(eXtensible Application Markup 
    Language)} Lenguaje basado en XML creado por \emph{Microsoft} como una 
    alternativa a código de programación para la creación de instancias 
    e inicialización de objetos, y la organización de esos objetos en 
    jerarquías de elementos primarios y secundarios.
    % https://docs.microsoft.com/es-es/xamarin/xamarin-forms/xaml/xaml-basics/

    \item \textbf{\textit{Web Semántica}}: ``\textit{Extensión de la actual 
    web en la que a la información disponible se le otorga un significado 
    bien definido que permita a los ordenadores y las personas trabajar en 
    cooperación. Se basa en la idea de tener datos en la web definidos y 
    vinculados de modo que puedan usarse para un descubrimiento, 
    automatización y reutilización entre varias aplicaciones.}''
    % Hendler, James, Berners-Lee, Tim and Miller, Eric 
    %"Integrating Applications on the Semantic Web," 
    % Journal of the Institute of Electrical Engineers of Japan, 
    % Vol 122(10), October, 2002, p. 676-680.

\end{itemize}
