% ARPEGOS:  Automatized Roleplaying-game Profile Extensible Generator Ontology based System %
% Author : Alejandro Muñoz Del Álamo %
% Copyright 2019 %

% Section 1.1: Motivación %

\vspace*{\fill}
\epigraph{\textit{“En Arrswyd, la oscuridad ha consumido las estrellas en el cielo, 
dejando nada más que una sola ciudad en medio del olvido. Arrswyd es una nación amurallada 
donde permanece la última población del mundo. El día ha sido olvidado hace mucho tiempo, 
los miedos se hacen realidad, y entre las torcidas espirales, merodeando entre las sombras, 
hay horrores que se alimentan del miedo. El terror es todo lo que Arrswyd sabe ya que su gente 
es perseguida por criaturas y oprimida por el Señor inmortal, Yomon Ecuro. Algunos se animan a 
luchar contra el abismo y las maquinaciones del Ministerio y del Instituto, mientras que otros 
simplemente quieren vivir los últimos días del mundo hasta que se apague la última luz.”}}{\textit{Arrswyd, Regnum Ex Nihilo}}
\vspace*{\fill}

\newpage

\section{Motivación}
Los juegos de rol, a los que llamaremos \textbf{\textit{RPG}} (\textit{\textbf{R}ole-\textbf{P}laying \textbf{G}ame}) a 
partir de ahora, son juegos con una meta particular, que es \textit{“El objetivo de ambos tipos de juegos, computerizados y 
de otro tipo, es experimentar una serie de aventuras en un mundo imaginario, a través de un personaje avatar o un pequeño grupo de 
personajes cuyas habilidades y poderes crecen conforme el tiempo pasa.”} \autocite*{Adams2010}. 

Uno de los aspectos más importantes de los \textit{RPG} son los personajes, ya que estos son la manifestación de los jugadores 
dentro del universo del juego. Como dicen Ramos y Sueiro~\autocite*{Ramos-Villagrasa2010}, a diferencia del teatro, donde la elección de un personaje puede depender de las características 
físicas del intérprete, en los juegos de rol cualquier persona puede interpretar cualquier personaje, siempre que esté dentro de las 
posibilidades que ofrezca el juego en que se esté jugando. Los personajes deben definirse dentro de los limites establecidos por 
las reglas del juego, y en una historia como si se tratara de su biografía, explicando cómo ha sido la vida del personaje en 
la ambientación hasta el momento de comenzar la historia. A este proceso de definir un personaje 
para un jugador se conoce como \textbf{creación de personaje}. \medskip

Hay un amplio rango de posibilidades cuando tratamos la creación de personajes en diferentes mundos. En algunos juegos, el jugador 
sólo puede seleccionar algunas características predefinidas para el personaje, mientras que en otros juegos el usuario puede cambiar cada
elemento del avatar \autocite*{Isaksson2012}. Esto hace que la creación de personajes sea un proceso arduo y complejo, pues es necesario 
tener amplios conocimientos del juego para conocer todas las opciones de personalización disponibles para el jugador, y sus correspondientes 
características. \medskip

% Hablar del proceso de creación %


El autor del blog \textit{Ars Rolica}~\autocite*{ArsRolica} comenta que una de las novedades que se están popularizando más es la de los 
generadores de personajes, que son herramientas que permiten crear personajes de juegos sin necesidad de invertir grandes cantidades 
de tiempo, evitando posibles errores y deslices. Estos generadores pueden ser bastante útiles tanto para jugadores novatos que no 
conocen el sistema del juego, como para jugadores más experimentados que no puedan utilizar tiempo para documentarse completamente 
antes de crear su álter ego. \medskip

El objetivo de este proyecto consiste en diseñar e implementar un generador de personajes de juegos de rol de mesa, para lo que se 
perseguirán los siguientes subobjetivos:
\begin{itemize}

    \item Realizar un estudio del arte de los generadores de personaje existentes, que permita conocer el estado actual de las herramientas
    que se encuentran en uso en la actualidad.

    \item Desarrollar un sistema que permita trabajar con diferentes juegos de rol, sin tener que salir de la aplicación, siempre y 
    cuando se disponga de los ficheros que contengan la información de dichos juegos.

    \item Diseñar un esquema del sistema de información que permita a la aplicación poder adaptarse de la forma más completa posible a 
    cada juego, permitiendo profundizar de la misma manera en juegos de alta complejidad que en juegos sencillos.

    \item Crear una interfaz de usuario intuitiva, que facilite al usuario la interacción con la aplicación y ésta resulte cómoda y 
    agradable de utilizar.

\end{itemize}







