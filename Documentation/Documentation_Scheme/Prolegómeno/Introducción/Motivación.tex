% ARPEGOS:  Automatized Roleplaying-game Profile Extensible Generator Ontology based System %
% Author : Alejandro Muñoz Del Álamo %
% Copyright 2019 %

% Section 1.1: Motivación %
\section{Motivación}
El sector de los juegos de rol está creciendo a pasos 
agigantados. Hoy en día existe un extensísimo 
catálogo de juegos de rol, que varían en función de la 
ambientación del ``universo'' en el que nos sumergimos. Por 
ejemplo, \textit{Dragones y Mazmorras} utiliza una 
ambientación medieval con elementos fantásticos, tales como 
magia y dragones, mientras que \textit{Warhammer 40.000} 
está basado en un futuro distópico en el que se mezclan 
elementos de ciencia ficción con elementos de fantasía 
heroica. \medskip

Debido a la expansión de este sector, y gracias al avance 
de la tecnología, podemos encontrar un sinfín de aplicaciones 
que tratan de mejorar la experiencia de los jugadores, 
desde juegos de mesa que hacen uso de sistemas de 
reproducción (\textit{CD}, \textit{DVD}, 
\textit{Blu-Ray}) para hacer del juego una experiencia 
interactiva (\textit{Atmosfear}, \textit{Piratas
del Caribe: El Cofre del Hombre Muerto}, \textit{Trivial 
Pursuit}, \textit{Cluedo}), hasta juegos de rol completamente
virtualizados que nos permiten sumergirnos en ellos, contemplar
sus parajes, disfrutar de su ambientación, y en algunos casos, 
nos permiten participar con jugadores de cualquier parte del 
mundo a través de la red (\textit{World of Warcraft}, \textit{Diablo III}, 
\textit{Black Desert Online}). \medskip

Dentro de este amplio espectro de posibilidades, los juegos de rol 
tradicionales también disponen de aplicaciones que enriquecen el 
transcurso de las partidas, y que proveen prestaciones útiles para las 
mismas, como lanzadores de dados virtuales, generadores de mazmorras, 
contadores de iniciativa, compendios de habilidades, etc. \medskip

Entre todas las utilidades que encontramos, cabe destacar las aplicaciones 
conocidas como \textit{generadores de personaje}. Estas facilitan a los 
jugadores todo lo necesario para desarrollar rápidamente la información 
indispensable de un personaje para poder interpretarlo en una partida. \medskip

Por otro lado, puesto que la tecnología tiene un ritmo de avance trepidante, 
y más aún la informática, hoy en día podemos echar cuenta de servicios 
que simplifican tareas cuya ejecución resultaba impensable en el pasado.
\medskip

Uno de estos servicios es la Web Semántica, que se basa en la idea de 
describir el contenido, el significado y la relación 
de los datos, de manera que sea posible evaluarlos automáticamente por 
máquinas de procesamiento, como pueden ser los ordenadores o los 
\textit{smartphones}.
\medskip

El planteamiento del presente proyecto surge de la concepción de una 
aplicación informática destinada a ampliar las funcionalidades de 
los generadores de personaje, innovando en el desarrollo de la misma 
al aplicar tecnologías de Web Semántica.\medskip
