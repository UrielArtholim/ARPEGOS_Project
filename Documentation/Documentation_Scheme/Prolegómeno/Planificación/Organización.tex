% ARPEGOS:  Automatized Roleplaying-game Profile Extensible Generator Ontology based System %
% Author : Alejandro Muñoz Del Álamo %
% Copyright 2019 %

% Section 2.3: Organización %
\section{Organización}

\subsection{Agentes involucrados}

Las personas involucradas en la elaboración de este proyecto son:
\begin{itemize}

    \item Los directores (o tutores) del proyecto: La labor de los directores del proyecto consiste
    en revisar el estado del proyecto durante todo el proceso de desarrollo y contribuir con las ideas 
    que estimen oportunas para perfeccionarlo. 

    \item Los clientes, que son los potenciales usuarios finales de la aplicación. Ellos indican los 
    requerimientos que consideran necesarios para que el producto final sea atractivo para el público 
    objetivo (\textit{target}) del producto.
    
    \item El equipo de desarrollo, formado en este caso por una persona, que se encarga de la 
    elaboración del proyecto.   

\end{itemize}

\subsection{Roles}
En esta sección se van a abordar los diversos roles que forman parte del equipo de desarrollo de este proyecto:
\begin{itemize}
    \item \textbf{Business Analyst}: Es la persona que se encarga de definir las funcionalidades y características
    del producto, priorización y refinamiento del \textit{backlog} (lista de trabajos pendientes). \autocite*{businessAnalyst}

    \item \textbf{Solution Architect}: Es el encargado de garantizar la integridad técnica y la coherencia de la solución 
    del proyecto. En este caso se considera contar con uno pues se utiliza tecnología cuya implementación no ha sido probada 
    con la funcionalidad deseada para el proyecto actual. \autocite*{solutionArchitect}
    
    \item \textbf{UX Designer}: Es el responsable de elaborar el diseño de cada vista con la que interactúa el usuario. \autocite*{UXDesigner}
    
    \item \textbf{Software Developer}: Es la persona encargada de convertir la especificación del sistema a código funcional. \autocite*{Developer}
    
    \item \textbf{Tester}: Es el encargado del apartado de pruebas del proyecto, desde la identificación de las condiciones de las pruebas, 
    hasta su automatización, pasando por su creación y la especificación de los procesos de prueba. \autocite*{Tester} 
    
    \item \textbf{Documentalist}: Es la persona especializada en ayudar a investigadores y desarrolladores en su búsqueda de 
    documentación científica y/o técnica. \autocite*{Documentalist}
\end{itemize}

\subsection{Recursos utilizados}
Los recursos que han sido utilizados en el desarrollo de la aplicación son:
\begin{itemize}
    \item Un ordenador personal, en el que se ha constituido el entorno de trabajo.
    \item Visual Studio 2019 Community, como entorno de desarrollo para la aplicación.
    \item Protégé, un editor de ontologías de código abierto
    \item Xamarin, como plataforma para el desarrollo de aplicaciones multiplataforma.
    \item LaTeX, como herramienta para la elaboración de la memoria.
    \item Dispositivo móvil con \textit{Android}, para probar que la aplicación funciona en la plataforma deseada.
\end{itemize}

Como todos los recursos software previamente citados son gratuitos, su uso no supone coste 
alguno en licencias de aplicaciones. No se tendrá en cuenta el precio del ordenador personal en 
los costes del proyecto, ya que el equipo de desarrollo disponía de uno con las capacidades 
necesarias para poner en funcionamiento los recursos software.

