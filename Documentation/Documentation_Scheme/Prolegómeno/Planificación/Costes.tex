% ARPEGOS:  Automatized Roleplaying-game Profile Extensible Generator Ontology based System %
% Author : Alejandro Muñoz Del Álamo %
% Copyright 2019 %

% Section 2.4: Costes %

\section{Costes}
\subsection{Costes humanos}

Para calcular los costes humanos, debemos tener alguna referencia salarial de un desarrollador, 
utilizándolo como base para realizar los cálculos en tiempo de desarrollo. Hemos considerado como 
una referencia aceptable las \textit{tablas salariales para personal investigador encargado del desarrollo 
de proyectos de investigación científica o técnica a través de un contrato por obra y servicio}.
Estas tablas indican que el \textit{coste anual total} de un ingeniero son 27.664,14 \euro, que al 
dividirse en 14 pagas (12 meses y 2 pagas extra), resulta que el coste mensual sería de 2.305,35 \euro. \medskip

La duración del proyecto ha sido de ``introducir número aquí`` días, resultando en 12 meses aproximadamente.
A continuación se muestra una tabla comparativa entre los costes estimados y los costes reales:

\begin{table}[H]
    \centering
    \begin{tabular}{|c|c|c|}
        \hline 
        & Tiempo (días) & Coste \\
        \hline \hline
        Estimado & & \\ \hline 
        Real & & \\ \hline 
    \end{tabular}
    \caption{Comparativa entre coste humano estimado y coste humano real}
\end{table}


\subsection{Costes materiales}
Como los dispositivos empleados son propiedad del equipo de desarrollo y el software utilizado 
en el proceso de elaboración del proyecto es gratuito, se asume un coste de 0 \euro{} en cuanto a costes materiales
se refiere.
