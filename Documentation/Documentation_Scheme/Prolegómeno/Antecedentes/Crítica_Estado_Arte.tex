% ARPEGOS:  Automatized Roleplaying-game Profile Extensible Generator Ontology based System %
% Author : Alejandro Muñoz Del Álamo %
% Copyright 2019 %

% Section 2.2: Crítica al estado del arte %

\section{Crítica al estado del arte} \label{Critica_Estado_Arte}
Tal y como se ha comentado previamente, existe un holgado abanico de 
aplicaciones cuya meta es mejorar y/o simplificar aspectos en lo referente a 
los juegos de rol, y aunque cumplen con su propósito, a veces no resultan 
tan efectivas como debieran. 
\medskip 

Esto puede deberse a que tras dedicar el tiempo y esfuerzo necesarios para 
desarrollar la aplicación, el estudio del juego ha aprovechado ese tiempo 
de producción para revisar el juego y editarlo, realizando modificaciones 
que provocan que \emph{\textbf{la aplicación quede obsoleta en poco tiempo}}. 
Como las aplicaciones suelen estar orientadas a una versión específica de un juego en concreto, 
éstas dejan de ser útiles cuando se desea utilizar una versión diferente del juego, o un juego 
diferente, lo que resulta en un mercado muy amplio de aplicaciones muy específicas, mientras que 
es muy difícil, casi imposible incluso, encontrar una que dé soporte a varios juegos a la vez, 
o a diferentes versiones del mismo juego. \medskip

Otro inconveniente es que las aplicaciones que requieren mucha información 
específica, como los generadores de personaje, pueden llegar a 
\emph{\textbf{resultar muy complejas}}, y al tener una interfaz poco intuitiva, 
provoca que el usuario no experimentado considere que el esfuerzo que tiene que 
dedicar para aprender cómo utilizarla es mayor que el de realizar el proceso manualmente. \medskip

También existen aplicaciones que no contemplan la reutilización de la 
información que han producido para efectuar operaciones que mejoren 
la jugabilidad, por lo que el usuario no le ve provecho a emplear dichas
aplicaciones.