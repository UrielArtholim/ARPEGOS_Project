% Project Management Plan Documentation Template %
% Template made following ISO/IEC/IEEE 16326:2009 %

% Author : Alejandro Muñoz Del Álamo %
% Copyright 2019 %

% Project Overview File %


\part{Prolegómeno}

\chapter{Introducción}
\thispagestyle{chapterpage}

\section{Motivación}
El sector de los juegos de rol está creciendo a pasos 
agigantados. Hoy en día existe un extensísimo 
catálogo de juegos de rol, que varían en función de la 
ambientación del ``universo'' en el que nos sumergimos. Por 
ejemplo, \textit{Dragones y Mazmorras} utiliza una 
ambientación medieval con elementos fantásticos, tales como 
magia y dragones, mientras que \textit{Warhammer 40.000} 
está basado en un futuro distópico en el que se mezclan 
elementos de ciencia ficción con elementos de fantasía 
heroica. \medskip

Debido a la expansión de este sector, y gracias al avance 
de la tecnología, podemos encontrar un sinfín de aplicaciones 
que tratan de mejorar la experiencia de los jugadores, 
desde juegos de mesa que hacen uso de sistemas de 
reproducción (\textit{CD}, \textit{DVD}, 
\textit{Blu-Ray}) para hacer del juego una experiencia 
interactiva (\textit{Atmosfear}, \textit{Piratas
del Caribe: El Cofre del Hombre Muerto}, \textit{Trivial 
Pursuit}, \textit{Cluedo}), hasta juegos de rol completamente
virtualizados que nos permiten sumergirnos en ellos, contemplar
sus parajes, disfrutar de su ambientación, y en algunos casos, 
nos permiten participar con jugadores de cualquier parte del 
mundo a través de la red (\textit{World of Warcraft}, \textit{Diablo III}, 
\textit{Black Desert Online}). \medskip

Dentro de este amplio espectro de posibilidades, los juegos de rol 
tradicionales también disponen de aplicaciones que enriquecen el 
transcurso de las partidas, y que proveen prestaciones útiles para las 
mismas, como lanzadores de dados virtuales, generadores de mazmorras, 
contadores de iniciativa, compendios de habilidades, etc. \medskip

Entre todas las utilidades que encontramos, cabe destacar las aplicaciones 
conocidas como \textit{generadores de personaje}. Estas facilitan a los 
jugadores todo lo necesario para desarrollar rápidamente la información 
indispensable de un personaje para poder interpretarlo en una partida. \medskip

Por otro lado, puesto que la tecnología tiene un ritmo de avance trepidante, 
y más aún la informática, hoy en día podemos echar cuenta de servicios 
que simplifican tareas cuya ejecución resultaba impensable en el pasado.
\medskip

Uno de estos servicios es la Web Semántica, que se basa en la idea de 
describir el contenido, el significado y la relación 
de los datos, de manera que sea posible evaluarlos automáticamente por 
máquinas de procesamiento, como pueden ser los ordenadores o los 
\textit{smartphones}.
\medskip

El planteamiento del presente proyecto surge de la concepción de una 
aplicación informática destinada a ampliar las funcionalidades de 
los generadores de personaje, innovando en el desarrollo de la misma 
al aplicar tecnologías de Web Semántica.\medskip


\section{Alcance}
El alcance de este proyecto comprende el desarrollo de una 
aplicación móvil para generar perfiles de personajes que 
puedan formar parte de una partida o campaña de un juego 
de rol, cuya base de datos esté incluida en la aplicación. \medskip

No forma parte del alcance de este proyecto el desarrollo 
de la base de datos del juego de rol, aunque ha sido 
necesario para poder realizar la sección de pruebas del
proyecto.

\section{Objetivos}
El sistema va a tener tres finalidades claramente diferenciadas:
\begin{itemize}

    \item \textbf{Selección de juego}:La aplicación podrá dar 
    acceso a varios juegos, de manera que se pueda alternar 
    entre éstos, permitiendo utilizar el mismo mecanismo para 
    el catálogo de juegos disponible.

    \item \textbf{Creación/Modificación de personaje}: La 
    aplicación permitirá al usuario seleccionar la información
    necesaria para la creación de un personaje a su gusto 
    mediante un proceso guiado paso a paso. En caso de que el 
    personaje ya exista, permitirá al usuario visualizar y realizar 
    modificaciones a la información mostrada.

    \item \textbf{Automatización del proceso de cálculo 
    de habilidades}: La aplicación facilitará al usuario una 
    interfaz en la que, al indicar la habilidad que desea 
    utilizar, y el resultado de su lanzamiento de dados, se 
    devuelva el resultado total que se debe aplicar en el 
    combate. También podrá calcular el resultado de tiradas 
    enfrentadas.

\end{itemize}

Además, la aplicación deberá hacer muestra de las siguientes 
cualidades: 
\begin{itemize}

    \item \textbf{Generalidad}: La aplicación no estará directamente 
    vinculada a la información específica de un juego, de forma que 
    sea posible procesar diferentes bancos de datos, y por tanto, 
    se pueda utilizar la misma aplicación para varias versiones 
    distintas del mismo juego, o incluso para juegos completamente 
    diferentes.

    \item \textbf{Sencillez}: La aplicación dispondrá de una interfaz 
    simple y agradable, que permita al usuario hacer uso de sus 
    funciones de forma asequible, sea cual fuere la complejidad del 
    juego seleccionado.

\end{itemize}


% Plantilla Tabla Requisitos Funcionales
%\begin{changemargin}{-1cm}{-1cm}
%    \begin{table}[h] 
%        \centering
%        \begin{tabular}{|l|m{10cm}|}
%            \hline
%            [Código objetivo] & [Nombre Objetivo] \\ 
%            \hline
%            \hline
%            Descripción & [Descripción Objetivo] \\
%            \hline
%            Comentarios & [Comentarios Objetivo] \\
%            \hline
%        \end{tabular}
%        \caption{Objetivo [Nº]. [\textit{Nombre Objetivo}]}
%    \end{table}
%\end{changemargin}

\section{Glosario de Términos}
\begin{itemize}

    % Añadir WebSemántica

    \item \textbf{\textit{Android}}: Sistema operativo que se emplea 
    en dispositivos móviles, por lo general con pantalla táctil. 
    De este modo, es posible encontrar tabletas, teléfonos móviles 
    y relojes equipados con Android, aunque el software también 
    se usa en otros dispositivos. 
    % https://definicion.de/android/ 

    \item \textbf{\textit{Blazegraph}}: Base de datos de grafos de 
    código abierto, escalable y de alto rendimiento basada en 
    estándares. Escrito completamente en \emph{Java}, la plataforma soporta 
    las familias de especificaciones \emph{Blueprint} y \textbf{RDF}/\textbf{SPARQL 1.1}
    incluyendo consultas, actualizaciones, consultas federadas básicas
    y descripción de servicios. 
    % https://wiki.blazegraph.com/wiki/index.php/About_Blazegraph
    
    \item \textbf{\textit{C\#}}: Lenguaje de programación multiparadigma 
    desarrollado y estandarizado por Microsoft como parte de su 
    plataforma \textbf{\net}, que después fue aprobado como un estándar por la
    ECMA (\emph{ECMA-334}) e ISO (\emph{ISO/IEC 23270}). 
    C\# es uno de los lenguajes de programación diseñados para la 
    infraestructura de lenguaje común. Su sintaxis básica deriva de 
    \emph{C/C++} y utiliza el modelo de objetos de la plataforma \\net, 
    similar al de \emph{Java}, aunque incluye mejoras derivadas de otros lenguajes.
    % https://es.wikipedia.org/wiki/C_Sharp


    \item \textbf{\textit{RDFSharp}}: \emph{Framework} de código 
    abierto \textbf{C\#} diseñado para facilitar la creación de 
    aplicaciones \emph{\net} basadas en el modelo \textbf{RDF}, 
    que representa una solución didáctica directa para comenzar 
    a trabajar con conceptos de \emph{Semántica Web}.
    % https://www.w3.org/2001/sw/wiki/RDFSharp

    \item \textbf{\textit{Git}}: Software de control de versiones 
    diseñado por \emph{Linus Torvalds}, pensando en la eficiencia 
    y la confiabilidad del mantenimiento de versiones de aplicaciones 
    cuando éstas tienen un gran número de archivos de código fuente. 
    Su propósito es llevar registro de los cambios en archivos de 
    computadora y coordinar el trabajo que varias personas realizan 
    sobre archivos compartidos.
    % https://es.wikipedia.org/wiki/Git

    \item \textbf{\textit{GitHub}}: GitHub es una plataforma de desarrollo 
    colaborativo de software para alojar proyectos utilizando el sistema de 
    control de versiones \textbf{Git}.
    % https://conociendogithub.readthedocs.io/en/latest/data/introduccion/

    \item \textbf{\textit{IDE}}: \emph{(Integrated Development Environment)} 
    Aplicación con numerosas características que se pueden usar para muchos 
    aspectos del desarrollo de software.
    %  https://docs.microsoft.com/es-es/visualstudio/get-started/visual-studio-ide?view=vs-2019

    \item \textbf{\textit{iOS}}: Sistema operativo móvil de la multinacional
    \emph{Apple Inc}. Originalmente desarrollado para el iPhone (iPhone OS), 
    después se ha usado en dispositivos como el iPod touch y el iPad. 
    No permite su instalación en hardware de terceros.
    % https://es.wikipedia.org/wiki/IOS
    
    \item \textbf{\textit{Metodología Ágil}}: Las metodologías ágiles son 
    métodos de desarrollo de software en los que las necesidades y soluciones 
    evolucionan a través de una colaboración estrecha entre equipos 
    multidisciplinarios. Se caracterizan por enfatizar la comunicación 
    frente a la documentación, por el desarrollo evolutivo y por su 
    flexibilidad.
    % https://es.wikiversity.org/wiki/Metodolog%C3%ADas_%C3%A1giles_de_desarrollo_software
    
    \item \textbf{\textit{Modelo}}: Las clases de modelo son clases no 
    visuales que encapsulan los datos de la aplicación. Por lo tanto, se 
    puede considerar que el modelo representa el modelo de dominio de la 
    aplicación, que normalmente incluye un modelo de datos junto con la 
    lógica de validación y negocios. 
    % https://docs.microsoft.com/es-es/xamarin/xamarin-forms/enterprise-application-patterns/mvvm

    \item \textbf{\textit{Modelo de Vista}}: El modelo de vista implementa las 
    propiedades y los comandos a los que la vista puede enlazarse y notifica 
    a la vista de cualquier cambio de estado a través de los eventos de 
    notificación de cambios. Las propiedades y los comandos que proporciona 
    el modelo de vista definen la funcionalidad que ofrece la interfaz de 
    usuario, pero la vista determina cómo se mostrará esa funcionalidad.
    % https://docs.microsoft.com/es-es/xamarin/xamarin-forms/enterprise-application-patterns/mvvm
    
    \item \textbf{\textit{MVVM}}: Patrón de arquitectura de software que
    ayuda a separar la lógica de negocios y presentación de una aplicación 
    de su interfaz de usuario. 
    % https://docs.microsoft.com/es-es/xamarin/xamarin-forms/enterprise-application-patterns/mvvm

    \item \textbf{\textit{Ontología}}:  Definición formal de tipos, propiedades, 
    y relaciones entre entidades que realmente o fundamentalmente existen 
    para un dominio de discusión en particular. Es una aplicación práctica 
    de la ontología filosófica, con una taxonomía.
    %  https://es.wikipedia.org/wiki/Ontolog%C3%ADa_(inform%C3%A1tica)

    \item \textbf{\textit{OWL}}: \emph{(Ontology Web Language)} es un lenguaje 
    de marcado semántico para publicar y compartir ontologías en la 
    World Wide Web. OWL se desarrolla como una extensión de vocabulario 
    de \textbf{RDF} y es derivado del lenguaje DAML + OIL asi.
    % https://www.w3.org/TR/owl-ref/


    \item \textbf{\textit{Protégé}}: Framework editor de ontologías de 
    código abierto y gratuito para construir sistemas inteligentes.
    % https://protege.stanford.edu/

    \item \textbf{\textit{RDF}}: \emph{(Resource Description Framework)} 
    Modelo estándar para el intercambio de datos en la Web. RDF 
    tiene características que facilitan la fusión de datos incluso si 
    los esquemas subyacentes difieren, y admite específicamente la 
    evolución de los esquemas a lo largo del tiempo sin requerir que 
    se cambien todos los consumidores de datos.
    % https://www.w3.org/RDF/

    \item \textbf{\textit{RDFS}}: \emph{RDF Schema} es una 
    extensión del vocabulario básico de \emph{RDF} que proporciona un 
    vocabulario de modelado de datos para los datos relativos a este modelo.
    % https://www.w3.org/TR/rdf-schema/

    \item \textbf{\textit{Scrum}}: Marco de trabajo para la gestión y 
    desarrollo del software basada en un proceso iterativo e incremental 
    utilizado comúnmente en entornos basados en el desarrollo ágil del
    software.
    % Carmen Lasa Gómez Alonso Alvarez García, Rafael de las Heras del Dedo. 
    % "Métodos Ágiles y Scrum". Anaya Multimedia, 2012.
    
    \item \textbf{\textit{SonarQube}}: Herramienta de revisión automática 
    de código para detectar errores, vulnerabilidades y olores de código 
    en su código. Se puede integrar con su flujo de trabajo existente 
    para permitir la inspección continua de código en todas las ramas 
    de su proyecto y solicitudes de extracción.
    % https://docs.sonarqube.org/latest/

    \item \textbf{\textit{SPARQL}}: \emph{(SPARQL Protocol and RDF Query 
    Language)} es un lenguaje y protocolo de consulta para \emph{RDF}. 
    % https://www.w3.org/TR/rdf-sparql-protocol/

    \item \textbf{\textit{Sprint}}: Período en el cual se lleva el 
    desarrollo de una tarea.
    % Carmen Lasa Gómez Alonso Alvarez García, Rafael de las Heras del Dedo. 
    % "Métodos Ágiles y Scrum". Anaya Multimedia, 2012.

    \item \textbf{\textit{Tarsier}}: Herramienta para la visualización 
    interactiva en 3D de grafos RDF.\@
    % Fabio Viola, Luca Roffia, Francesco Antoniazzi, Alfredo D’Elia, Cristiano Aguzzi and Tullio Salmon Cinotti     
    % "Interactive 3D Exploration of RDF Graphs through Semantic Planes" 17 August 2018

    \item \textbf{\textit{UML}}: \emph{(Unified Modeling Language)} es el 
    lenguaje de modelado de sistemas de software más conocido y utilizado 
    en la actualidad.
    % https://es.wikipedia.org/wiki/Lenguaje_unificado_de_modelado

    \item \textbf{\textit{Vista}}: La vista es responsable de definir 
    la estructura, el diseño y la apariencia de lo que el usuario ve 
    en la pantalla. Idealmente, cada vista se define en XAML, con un 
    código subyacente limitado que no contiene la lógica de negocios. 
    Sin embargo, en algunos casos, el código subyacente podría contener 
    lógica de la interfaz de usuario que implementa el comportamiento 
    visual que es difícil de expresar en XAML, como animaciones.
    % https://docs.microsoft.com/es-es/xamarin/xamarin-forms/enterprise-application-patterns/mvvm

    \item \textbf{\textit{Visual Studio}}: El \emph{IDE} de Visual Studio 
    es un panel de inicio creativo que se puede usar para editar, 
    depurar y compilar código y, después, publicar una aplicación.
    Más allá del editor estándar y el depurador que proporcionan la 
    mayoría de \emph{IDE}, Visual Studio incluye compiladores, herramientas 
    de finalización de código, diseñadores gráficos y muchas más 
    características para facilitar el proceso de desarrollo de software.
    % https://docs.microsoft.com/es-es/visualstudio/get-started/visual-studio-ide?view=vs-2019

    \item \textbf{\textit{W3C}}: \emph{(World Wide Web Consortium)} Consorcio 
    internacional que produce recomendaciones para la \emph{WWW}.
    % https://www.w3c.es/Consorcio/

    \item \textbf{\textit{WWW}}: \emph{(World Wide Web)} Sistema de distribución 
    de información basado en hipertexto o hipermedios enlazados y accesibles 
    a través de Internet. 
    % https://es.wikipedia.org/wiki/World_Wide_Web

    \item \textbf{\textit{Xamarin}}: Xamarin es una plataforma de código 
    abierto para compilar aplicaciones modernas y de rendimiento para iOS, 
    Android y Windows con \net.
    % https://docs.microsoft.com/es-es/xamarin/get-started/what-is-xamarin

    \item \textbf{\textit{Xamarin.Forms}}: Xamarin.Forms es un marco de 
    interfaz de usuario de código abierto. Xamarin.Forms permite a los 
    desarrolladores compilar aplicaciones de Android, iOS y Windows 
    desde un único código base compartido.
    % https://docs.microsoft.com/es-es/xamarin/get-started/what-is-xamarin-forms

    \item \textbf{\textit{XAML}}: \emph{(eXtensible Application Markup 
    Language)} Lenguaje basado en XML creado por \emph{Microsoft} como una 
    alternativa a código de programación para la creación de instancias 
    e inicialización de objetos, y la organización de esos objetos en 
    jerarquías de elementos primarios y secundarios.
    % https://docs.microsoft.com/es-es/xamarin/xamarin-forms/xaml/xaml-basics/

    \item \textbf{\textit{Web Semántica}}: ``\textit{Extensión de la actual 
    web en la que a la información disponible se le otorga un significado 
    bien definido que permita a los ordenadores y las personas trabajar en 
    cooperación. Se basa en la idea de tener datos en la web definidos y 
    vinculados de modo que puedan usarse para un descubrimiento, 
    automatización y reutilización entre varias aplicaciones.}''
    % Hendler, James, Berners-Lee, Tim and Miller, Eric 
    %"Integrating Applications on the Semantic Web," 
    % Journal of the Institute of Electrical Engineers of Japan, 
    % Vol 122(10), October, 2002, p. 676-680.


\end{itemize}

\section{Metodología}
La aplicación objeto de este proyecto 
\lorem{} % To do

\section{Estructuración del documento}
El documento que se presenta está estructurado en una colección de capítulos, 
en los que se describen de manera precisa y detallada todas y cada una de 
las etapas por las que ha pasado el proyecto, desde su inicio hasta su 
conclusión. A continuación se muestra un breve resumen de los contenidos 
de cada capítulo:

\begin{itemize}
    \item \textbf{\emph{Capítulo 1. Introducción}}. El capítulo inicial 
    consiste en una introducción al proyecto, explicando los antecedentes 
    a su desarrollo, así como la motivación para ponerlo en práctica, los 
    objetivos que debe cumplir y un glosario de términos para facilitar 
    la comprensión del presente documento.

    \item \textbf{\emph{Capítulo 2. Conceptos básicos}}. Tras la introducción,
    se abordan los fundamentos necesarios para simplificar la lectura de 
    este documento. A su vez, se exponen las diferentes tecnologías de las 
    que se han aplicado para la consecución y puesta en marcha de este 
    proyecto.

    \item \textbf{\emph{Capítulo 3. Planificación del proyecto}}. Este 
    capítulo engloba toda la información relevante a aspectos de 
    suma importancia tales como la evaluación de riesgos o la planificación 
    temporal del proyecto.

    \item \textbf{\emph{Capítulo 4-\@8. Análisis, Diseño, Codificación y 
    Pruebas}}. Este conjunto de capítulos profundizan en los diversos 
    aspectos y fases que comprenden el desarrollo de un producto haciendo 
    uso de una \emph{metodología ágil}. Esto posibilita analizar y 
    enriquecer el producto en su desarrollo, mejorando así el resultado 
    final.

    \item \textbf{\emph{Capítulo 9. Manual de Instalación}}. Aquí se 
    ha desarrollado un documento con las instrucciones necesarias 
    para realizar la instalación de la aplicación.

    \item \textbf{\emph{Capítulo 10. Manual de Usuario}}. Se ha redactado 
    un manual de usuario para la aplicación, cuyo objetivo es garantizar un 
    uso eficiente y responsable de la misma por parte de los usuarios. 
    
    \item \textbf{\emph{Capítulo 11. Conclusiones}}. El último capítulo es 
    un breve repaso sobre el desarrollo del proyecto, que incluye una 
    opinión personal y un apartado de posibles mejoras que se podrían 
    realizar al proyecto en un futuro.

    % Incluir Apéndices

\end{itemize}

%\lorem{} % To do 







